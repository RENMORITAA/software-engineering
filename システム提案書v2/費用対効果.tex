\subsection{効果}
本システムの導入により、地域住民は多様な商品を手軽かつ迅速に購入できるようになる。
香美市の個人経営店舗を支援し、地域経済の活性化を促進することを目的としている。
ユーザーはスマートフォンを活用し、時間や場所を問わず商品を購入できるため、利便性が大幅に向上する。
店舗側もオンライン販売の拡大により、売上増加が期待できる。
また、システムは高齢者にも使いやすいユーザーフレンドリーなインターフェースを備えており、幅広い世代が安心して利用できる。
さらに、大学生のアルバイト機会も創出し、地域の若者の雇用促進と社会活性化にも貢献する。



\subsection{収益}
本システムの収益は、Uber Eatsの収益構造を参考にしており、
「店舗手数料」「配達料金の一部」「ユーザーサービス料」の3つを主な柱としている。
登録店舗は、アプリ上で注文が発生するたびに、注文金額の30%を手数料として運営側に支払う。
平均お弁当価格を700円とした場合、1件あたりの手数料収益は210円となる。
ユーザーは配達距離に応じて「基本料金150円+距離50円/km」を支払う。
平均配達距離を2kmとすると配達料金は250円となり、そのうち30%(75円)を運営側収益として受け取る。
残りの70%は配達員報酬とする。
各注文につき、70円の固定手数料をユーザーから徴収し、これを全額収益とする。

年間件(1日あたり約220件)の注文を想定すると、
\[
年間総収益=355円 \times 約80,000件 = 28,400,000円 となる。
\]
これを5年間運用すると
 \[
 28,400,000\text{ 円} \times 5\text{ 年} = 142,000,000\text{ 円}
 \]

\subsection{費用}
本システムの主な費用は、サーバー運用・開発保守・プロモーション・配達員報酬などである。
以下に年間費用をまとめる。
\begin{table}[H]
\centering
\caption{初期費用}
\begin{tabular}{|c|c|c|c|}
\hline
項目 & 単価 (円) & 数量 & 金額 (円) \\
\hline
管理者用端末 & 100,000 & 6 & 600,000 \\
\hline
\end{tabular}
\end{table}

また、運用費用は以下の通りとなる。

\begin{table}[H]
\centering
\caption{運用費用}
\begin{tabular}{|c|c|c|c|}
\hline
項目 & 単価 (円) & 数量 & 金額 (円) \\
\hline
人件費 & 4,000,000 / 年 & 6 & 24,000,000 \\
\hline
サーバーレンタル & 36,000 / 年 & 1 & 36,000 \\
\hline
\end{tabular}
\end{table}

よって、5年間の費用合計は以下のようになる。

\[
(24,000,000\text{ 円} + 36,000\text{ 円}) \times 5 = 120,180,000\text{ 円}
\]
\[
120,180,000\text{ 円} + 600,000\text{ 円} = \boxed{120,780,000\text{ 円}}
\]

\subsection{利益}
このアプリを5年間運用した利益は以下のようになる
\[
142,000,000\text{ 円}-120,780,000\text{ 円} = 21,220,000\text{ 円}
\]