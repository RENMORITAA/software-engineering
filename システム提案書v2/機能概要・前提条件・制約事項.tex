\subsection{機能概要}

\subsubsection{配達側}

\begin{itemize}
  \item 会員機能
    \begin{itemize}
        \item 新規会員機能
        
        名前、性別、歳、住所、メールアドレス、電話番号、パスワードを登録し会員になる機能。また、新規会員登録の際に規約を提示し、規約に従う場合に新規登録を行う。ただし、パスワードは作成条件を つける。
        \item ログイン・ログアウト
        
        ログイン機能では、すでに会員である配達側がメールアドレス、パスワードを入力することによっ てアプリにログインをする。ログアウト機能では、アプリにログインしている会員が任意のタイミング でログアウトできる。パスワードを忘れた際などに、メールアドレスを入力することによってパスワー ドを再設定することが可能。
        \item 退会
        
        会員が任意のタイミングでアプリ会員から退会する機能。会員情報を完全に削除する。
        \item 会員情報確認
        
        会員ごとに存在するマイページを確認する機能。会員情報として登録している情報が表示される。
        \item 会員情報変更
        
        登録している会員情報を変更する機能。
        \item 履歴書情報記述機能
        
        求人に応募する場合に用いる履歴書を記述する機能。この機能は新規会員登録の場合にはスキップができ、後々記述することが可能。
        \item 問い合わせ機能
        
        この画面ではアプリの運営への問い合わせができ、操作方法等の質問をメール形式で送信することが可 能となっている。質問への回答はアプリ利用者が登録したメールアドレスへ送信される。
        \item パスワード変更機能
      
        パスワードを作成条件下で任意に変更する機能。
        \item 配達先決定機能
        注文を受けてから配達先を決定する機能。地図上で配達先の位置を確認し、配達先を決定する。
        
    \end{itemize}
  \item メディア機能
  \begin{itemize}
      \item 求人情報表示
      
      香美市内にある求人情報を表示する。一覧画面では、店舗名、店舗の住所、配達先の住所、配達物の情報、報酬額を表示する。一覧に表示する順序は、新着順で行う。

      求人情報の表示の際に、注意事項を表示する。注意事項は、台風や悪天候等の場合、自己判断にて配達可能ならば応募を受注してください。と明記する。

     %配達手段として、自転車以上が推奨のような表示があるといいかも
      
      \item タグ分け機能
      
      タグは地域(土佐山田、香北など)、商品のジャンル、配達希望時間、金額とする。タグ分け機能では、それぞれのタグを指定した際にそのタグ内容ごとに求人の表示を行う機能である。
      
      \item 求人検索機能
      
      求人情報一覧より、利用者が任意のワードを入力することでそのワードに該当する求人が表示される。
    
      
      \item 配達履歴機能
      
      配達側ユーザーが配達を完了した依頼の履歴を取得しておき、マイページにて閲覧する機 能。閲覧履歴ページにて情報を選択すると詳細ページへと進む。詳細ページでは店舗名、配達物の情報、報酬額、受注時刻、完了時刻を表示する。配達側ユーザーが配達を完了した依頼の履歴を取得しておき、マイページにて閲覧する機 能。閲覧履歴ページにて情報を選択すると詳細ページへと進む。詳細ページでは店舗名、配達物の情報、報酬額、受注時刻、完了時刻を表示する。
      \item 求人応募機能
      
      求人に対して応募を行う機能。このボタンを押すには履歴書情報をマイページより記入する必要があ り、記入していない状態で押すと履歴書情報記入ページへと進むこととする。ボタンを押すと店舗に対して配達ユーザーの履歴書情報、連絡先が送られ、店舗と配達ユーザー間の連絡 を行う。
  \end{itemize}
\end{itemize}


\subsubsection{依頼側}

\begin{itemize}
  \item 会員機能
    \begin{itemize}
        \item 新規会員機能
        
        名前、性別、歳、住所、メールアドレス、電話番号、パスワードを登録し会員になる機能。また、新規会員登録の際に規約を提示し、規約に従う場合に新規登録を行う。ただし、パスワードは作成条件を つける。
        \item ログイン・ログアウト
        
        ログイン機能では、すでに会員である配達側がメールアドレス、パスワードを入力することによっ てアプリにログインをする。ログアウト機能では、アプリにログインしている会員が任意のタイミング でログアウトできる。パスワードを忘れた際などに、メールアドレスを入力することによってパスワー ドを再設定することが可能。
        \item 退会
        
        会員が任意のタイミングでアプリ会員から退会する機能。会員情報を完全に削除する。
        \item 会員情報確認

        会員ごとに存在するマイページを確認する機能。会員情報として登録している情報が表示される。
        \item 会員情報変更
        
        登録している会員情報を変更する機能。
        \item 問い合わせ機能
        
        この画面ではアプリの運営への問い合わせができ、操作方法等の質問をメール形式で送信することが可 能となっている。質問への回答はアプリ利用者が登録したメールアドレスへ送信される。
        \item パスワード変更機能
        
        パスワードを作成条件下で任意に変更する機能。

    \end{itemize}

  \item メディア機能
    \begin{itemize}
        \item 店舗検索機能
        
        依頼者が任意のワードを入力し、そのワードに該当する店舗が表示・検索される。
        
        \item 商品検索機能
        
        依頼者が任意のワードを入力し、そのワードに該当する商品が表示・検索される。

        \item カート機能

        商品を選択すると一時的にカートに追加される。また、カート内にある商品の情報を閲覧することが可能である。そして、選択済みおよびカートに追加した商品を削除することが可能である。

        注文が完了された商品はカートから削除される。

        
        \item 注文確定機能
        
        カート内に追加された商品の注文を確定する。
        
        \item 配達先候補の住所登録・削除機能
        
        配達先として複数の住所を登録し、指定することが可能である。候補は3つまで登録・削除可能である。
        \item 配達状況確認機能
        
        配達状況は、依頼済み、配達受付完了、配達中、配達完了の4段階に分けられ、依頼者が現在の配達状況を確認できる。

        \item 配達員評価機能
        
        商品の配達完了後に、配達スピード・丁寧さ・対応の印象などに基づき、 配達員を5段階評価できる。

        また、配達員に不適切な言動や行動があった場合に、管理側に通報することができる。

        \item 店舗評価機能
        
        商品の配達完了後に、料理の味・梱包・対応などに基づき、店舗を5段階評価できる。
        
        \item 配達員評価閲覧機能 
        
        登録されている配達員の情報、およびユーザーからの評価を閲覧できる。

        \item 店舗評価閲覧機能 
        
        登録されている店舗の情報、およびユーザーからの評価を閲覧できる。 
    \end{itemize}
  
\end{itemize}



\subsubsection{店舗側}

\begin{itemize}
  \item 会員機能
    \begin{itemize}
        \item 店舗新規登録申し込み
        %住所入力の際に対応エリア内であるのかを判断してくれるといいな
        
        店舗新規登録申し込みでは店舗名、店舗所在地、飲食店営業許可書のファイルおよび写真、メールアドレスまたは電話番号を入力して申込を行う。

        \item 新規会員機能
        
        名前、性別、歳、住所、メールアドレス、電話番号、パスワードを登録し会員になる機能。また、新規会員登録の際に規約を提示し、規約に従う場合に新規登録を行う。ただし、パスワードは作成条件を つける。
        \item ログイン・ログアウト
        
        ログイン機能では、すでに会員である配達側がメールアドレス、パスワードを入力することによっ てアプリにログインをする。ログアウト機能では、アプリにログインしている会員が任意のタイミング でログアウトできる。パスワードを忘れた際などに、メールアドレスを入力することによってパスワー ドを再設定することが可能。
        \item 退会
        
        会員が任意のタイミングでアプリ会員から退会する機能。会員情報を完全に削除する。
        \item 会員情報確認
        
        会員ごとに存在するマイページを確認する機能。会員情報として登録している情報が表示される。
        \item 会員情報変更
        
        登録している会員情報を変更する機能。
        \item 問い合わせ機能
        
        この画面ではアプリの運営への問い合わせができ、操作方法等の質問をメール形式で送信することが可 能となっている。質問への回答はアプリ利用者が登録したメールアドレスへ送信される。
        \item パスワード変更機能
        
        パスワードを作成条件下で任意に変更する機能。
        \item メニュー情報登録機能
        
        商品名、対象の写真、価格、説明文、カテゴリ(野菜や肉など)を登録する機能。
        \item メニュー情報更新機能
        
        メニューの商品名、対象の写真、価格、説明文、カテゴリを更新する機能。
        \item メニュー情報削除機能
        
        登録されているメニューを任意のタイミングで削除する機能。

        \item 在庫更新機能
        
        メニューの在庫(販売中 or 売り切れ)を更新する機能。
    \end{itemize}
  
\end{itemize}

\subsubsection{管理者側}
%コピペで持ってきたから内容をいい感じに変える。

\begin{itemize}
  \item アカウント管理
    \begin{itemize}
        \item 依頼者アカウントの管理機能
        
        管理者は、依頼者のアカウントを登録・編集・削除することができる。また、商品の配達状況や注文履歴などの管理を行う。


        \item 配達員アカウント管理機能
        
        管理者は、配達員のアカウントを登録・編集・削除することが可能である。また、配達履歴の管理を行う。

        \item 店舗アカウント管理機能

        管理者は、店舗のアカウントを登録・編集・削除することが可能である。
        
        \item 管理者アカウントの管理

        他の管理者のアカウントを登録・編集・削除することが可能である。
        
        \item 注文管理機能
        
        管理者は、ユーザーからの問い合わせを受け取り表示することができる。

        \item 支払い額管理機能

        依頼者および店舗が支払う料金を管理する。また、配達員への報酬額を管理する。


    \end{itemize}
  
\end{itemize}

\subsection{前提条件}
 本システム提案書では以下の条件を前提条件とする。

 \begin{itemize}
  \item 利用者がインターネットに接続可能な端末(iOSまたはandroidに対応)を保有していること。
  \item 利用者が本アプリケーションの規約に同意済みであること。
  \item 本アプリケーションの使用には会員登録が必須であること。
\end{itemize}

店舗側は店舗申請を行う際、以下の出店条件を満たすこととする。

\begin{itemize}
  \item 対応エリア(香美市)内に店舗があること。
  \item 飲食店営業許可証があること。
\end{itemize}

\subsection{制約事項}
本システム提案書では以下の事項を制約事項とする。
\begin{itemize}
    \item 利用者の住所や氏名などの個人情報の漏洩を防ぐ仕様であること。
    \item 管理者側による投稿の削除・規制が行える仕様であること。
\end{itemize}
