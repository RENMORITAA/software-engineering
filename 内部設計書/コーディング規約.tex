本章では、Flutter の開発における Dart の命名規約およびコーディングスタイルを示す。

\subsection{ケース(表記形式)}

表\ref{table:case} に命令時に用いるケースの定義を示す。

\begin{table}[h]
\centering
\caption{命名時に用いるケースの定義}
\label{table:case}
\begin{tabular}{|c|c|c|}
\hline
ケース & 説明 & 例 \\
\hline
lowerCamelCase & 先頭小文字,単語区切り大文字 & userName \\
UpperCamelCase & 先頭大文字,単語区切り大文字 & UserProfile \\
snake\_case & 小文字,アンダースコアつなぎ & user\_profile \\
UPPER\_SNAKE\_CASE & 大文字,アンダースコアつなぎ & MAX\_COUNT \\
\hline
\end{tabular}
\end{table}

\subsection{Dart の命名規約}

表\ref{table:naming} に Dart における対象ごとの命名規約を示す。

\begin{table}[h]
\centering
\caption{Dart の命名規約}
\label{table:naming}
\begin{tabular}{|c|c|c|}
\hline
対象 & ケース & 例 \\
\hline
モジュール名/パッケージ名 & snake\_case & user\_profile, my\_package \\
ファイル名/ディレクトリ名 & snake\_case & user\_detail\_page.dart \\
クラス名 & UpperCamelCase & UserModel \\
メソッド/関数名 & lowerCamelCase & getUserData() \\
変数名 & lowerCamelCase & userCount \\
定数名 & lowerCamelCase & maxRetryCount \\
\hline
\end{tabular}
\end{table}

以下の点に注意する。

\begin{itemize}
  \item private でない識別子をアンダースコア(\_)から始めないこと。
  \item ハンガリアン記法(mUser, kUser など)は使用しない。
  \item 未使用コールバック引数には \_(または複数個の \_ )を使用する。
\end{itemize}


\subsection{コーディングスタイル}
Dart のコーディングスタイルは Effective Dart に従う。
以下に主な規約を記載する。

\subsubsection{基本スタイル}

\begin{itemize}
  \item インデントは2スペースとする。
  \item 空行にはインデントを入れない(スペースを入れない)。
  \item オブジェクト生成時に new は記述しない。const は必要な場合のみ使用する。
  \item 再代入しない変数は final として宣言する。
\end{itemize}

\subsubsection{Widget(Flutter)記述ルール}

\begin{itemize}
  \item 引数が2つ以上ある Widget は複数行で書く。
  \item Widget の引数が1つでも、その子 Widget が複数行の場合は親も改行する。
  \item Widget リストは末尾カンマ(trailing comma)をつけ、自動整形に従う。
  \item EdgeInsets は複数引数でも1行でよい。
\end{itemize}

\subsubsection{制御構文}

\begin{itemize}
  \item if, for, while などの制御構文は中括弧 \{ \} を省略しない。
  \item ただし、if と else がいずれも1行で完結する場合のみ、省略を許容する。
\end{itemize}

\subsubsection{コメント}

\begin{itemize}
  \item // の後には半角スペースを1つ置く。
  \item 意図が明確なコードに対して不要なコメントを記述しない。
\end{itemize}

\subsubsection{import / export 規約}

Dart における import 文は、読みやすさと依存関係の明確化のため、
以下の順序で記述するものとする。

\begin{enumerate}
  \item \textbf{dart:} 標準ライブラリ
  \item \textbf{package:} 外部パッケージ
  \item \textbf{package:(自プロジェクト)} アプリ内 lib/ を package import で参照
  \item \textbf{相対パス:} ../ や ./ を用いた参照
\end{enumerate}

その他の注意点:

\begin{itemize}
  \item import は辞書順で並べる。
  \item export は import とは別のグループとして最後にまとめる。
  \item 可能な限り相対 import ではなく package import を使用する。
\end{itemize}

\subsubsection{書式}

\begin{itemize}
  \item 原則として1行80文字以内とする。
  \item dart の自動 formatter に従う。
\end{itemize}

\subsection{ツール設定}

以下のツールを導入し、自動で規約を適用できるようにする。

\begin{itemize}
  \item VSCode の "Format on Save" を有効にする。
  \item \texttt{dart fix --apply} により自動修正を行う。
  \item \texttt{flutter analyze} でコーディング規約違反を確認する。
\end{itemize}
