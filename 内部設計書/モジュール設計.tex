以下に本アプリケーションのモジュール設計を示す。

\begin{itemize}
  \item overlay は、アプリケーション上で一番上のレイヤーに表示されるウィジェットをまとめたモジュールについて記述している。
  \item provider は、アプリケーション全体で使用する共通の変数をまとめたモジュールについて記述している。
  \item page は、アプリケーションの画面をまとめたモジュールについて記述している。
  \item component は、page で用いられるパーツをまとめたモジュールについて記述している。
\end{itemize}


\begin{table}[htbp]
\centering
\begin{tabularx}{\textwidth}{|l|X|}
\hline
\rowcolor[RGB]{200,200,200}
\multicolumn{2}{|c|}{\textbf{overlay モジュール}} \\
\hline
\textbf{モジュール名} & \textbf{概要} \\
\hline
\multicolumn{2}{|l|}{\textbf{<共通機能>}} \\
\hline
logout.dart & ログアウトを確定するかどうかを確認するオーバーレイ \\
withdraw.dart & 退会を確定するかどうかを確認するオーバーレイ \\
cancel\_inf\_change.dart & 情報の変更をキャンセルするかを確認するオーバーレイ \\
confirm\_inf\_change.dart & 情報の変更を確定するかを確認するオーバーレイ \\
rule\_screen.dart & 利用規約を表示するオーバーレイ \\
\hline
\multicolumn{2}{|l|}{\textbf{<依頼側>}} \\
\hline
\multicolumn{2}{|l|}{\textbf{<配達側>}} \\
\hline
d\_payslip\_change\_date.dart & 配達員の給与明細の表示される期間を変更するモジュール \\
\hline
\multicolumn{2}{|l|}{\textbf{<店舗側>}} \\
\hline
\end{tabularx}
\caption{overlay モジュールの構成}
\end{table}


\begin{table}[htbp]
\centering
\begin{tabularx}{\textwidth}{|l|X|}
\hline
\rowcolor[RGB]{200,200,200}
\multicolumn{2}{|c|}{\textbf{provider モジュール}} \\
\hline
\textbf{モジュール名} & \textbf{概要} \\
\hline
change\_user\_role.dart & アプリ全体で使用するプロバイダモジュール \\
app\_router.gr.dart & RouterをRouterに変換するモジュール \\
\hline
\end{tabularx}
\caption{provider モジュールの構成}
\end{table}


\begin{table}[htbp]
\centering
\footnotesize % ここでフォントサイズを小さくする (\small や \scriptsize も選択肢)
\begin{tabularx}{\textwidth}{|l|X|}
\hline
\rowcolor[RGB]{200,200,200}
\multicolumn{2}{|c|}{\textbf{page モジュール}} \\
\hline
\textbf{モジュール名} & \textbf{概要} \\
\hline
\multicolumn{2}{|l|}{\textbf{<共通機能>}} \\
\hline
change\_root\_page.dart & アプリ全体を包括するモジュール \\
new\_member.dart & 新規会員登録画面を生成するモジュール \\
mypage.dart & マイページ画面を表示するモジュール \\
login\_page.dart & ログイン画面を生成するモジュール \\
logout\_withdraw\_page.dart & ログアウト・退会画面を生成するモジュール \\
\hline
\multicolumn{2}{|l|}{\textbf{<依頼側>}} \\
\hline
c\_root\_page.dart & 「ホーム」、「カート」、「マップ」、「マイページ」、「通知」を包括するモジュール \\
c\_info.dart & 依頼者の会員情報画面を表示するモジュール \\
c\_info\_change.dart & 依頼者の会員情報変更画面を表示するモジュール \\
c\_address.dart & 配達先住所確認画面を表示するモジュール \\
c\_address\_change.dart & 配達先住所編集画面を表示するモジュール \\
c\_address\_add.dart & 配達先住所追加画面を表示するモジュール \\
c\_banking\_info.dart & 口座情報管理画面を表示するモジュール \\
c\_banking\_info\_change.dart & 口座情報変更画面を表示するモジュール \\
c\_banking\_info\_check.dart & 口座情報変更を確認する画面を表示するモジュール \\
c\_order\_history.dart & 注文履歴一覧画面を表示するモジュール \\
c\_order\_history\_detail.dart & 注文履歴詳細画面を表示するモジュール \\
c\_order\_arrival\_report.dart & 商品到着報告画面を表示するモジュール \\
c\_home.dart & 依頼者のホーム画面を表示するモジュール \\
c\_payment.dart & 支払い明細画面を表示するモジュール \\
c\_product\_list.dart & 商品一覧画面を表示するモジュール \\
c\_order\_check.dart & 注文確認画面を表示するモジュール \\
c\_order\_confirm.dart & 注文確定画面を表示するモジュール \\
c\_cart.dart & カート情報を表示するモジュール \\
c\_map\_search.dart & マップでの店舗検索を行う画面を表示するモジュール \\
\hline
\multicolumn{2}{|l|}{\textbf{<配達側>}} \\
\hline
d\_home.dart & ホーム画面を表示するモジュール \\
d\_notice.dart & 通知一覧画面を表示するモジュール \\
d\_notice\_detail.dart & 通知詳細画面を表示するモジュール \\
d\_job\_select.dart & 求人検索一覧画面を表示するモジュール \\
d\_job\_detail.dart & 求人情報の詳細画面を表示するモジュール \\
d\_map.dart & 求人情報より配達先ルートを示したマップを表示するモジュール \\
d\_banking\_information.dart & 口座情報を示した画面を表示するモジュール \\
d\_resume\_detail.dart & 履歴書情報を示した画面を表示するモジュール \\
d\_resume\_change.dart & 履歴書情報を変更する画面を表示するモジュール \\
d\_payslip.dart & 給与明細を示した画面を表示するモジュール \\
d\_delivery\_history.dart & 配達履歴の一覧を示した画面を表示するモジュール \\
d\_delivery\_history\_detail.dart & 配達履歴詳細を示した画面を表示するモジュール \\
d\_banking\_information\_change.dart & 口座情報変更を行う画面を表示するモジュール \\
d\_banking\_information\_check.dart & 口座情報の変更を確認する画面を表示するモジュール \\
d\_job\_map\_select.dart & マップにて配達員先住所(求人情報)を検索する画面を表示するモジュール \\
d\_root\_page.dart & 「ホーム」、「求人検索」、「マップ」、「受注」、「マイページ」を包括するモジュール \\
\hline
\multicolumn{2}{|l|}{\textbf{<店舗側>}} \\
\hline
s\_root\_page.dart & 「ホーム」、「業務」、「注文管理」、「売上」、「マイページ」を包括するモジュール \\
s\_info.dart & 店舗の会員情報画面を表示するモジュール \\
s\_info\_change.dart & 店舗の会員情報変更画面を表示するモジュール \\
s\_inventory\_status.dart & 店舗の在庫ステータス画面を表示するモジュール \\
s\_menu\_regist.dart & メニュー情報の新規登録画面を表示するモジュール \\
s\_menu\_change.dart & メニュー情報の変更画面を表示するモジュール \\
s\_menu\_delete.dart & メニュー情報の削除画面を表示するモジュール \\
s\_menu\_edit.dart & メニュー情報設定画面を表示するモジュール \\
s\_order\_list.dart & 受注一覧画面を表示するモジュール \\
s\_work.dart & 業務画面を表示するモジュール \\
s\_home.dart & 店舗ホーム画面を表示するモジュール \\
s\_delivery\_complete.dart & 受け渡し完了確認画面を表示するモジュール \\
s\_banking\_info.dart & 口座情報管理画面を表示するモジュール \\
\hline
\end{tabularx}
\caption{page モジュールの構成}
\end{table}


\begin{table}[htbp]
\centering
\begin{tabularx}{\textwidth}{|l|X|}
\hline
\rowcolor[RGB]{200,200,200}
\multicolumn{2}{|c|}{\textbf{component モジュール}} \\
\hline
\textbf{モジュール名} & \textbf{概要} \\
\hline
\multicolumn{2}{|l|}{\textbf{<共通機能>}} \\
\hline
finish\_screen.dart & 完了画面を生成するコンポーネント \\
over\_screen\_controller.dart & オーバーレイの表示、非表示の操作を行うコンポーネント \\
\hline
\multicolumn{2}{|l|}{\textbf{<依頼側>}} \\
\hline
c\_number\_count.dart & 注文数のカウントを増加(減少)するボタンを生成するレイヤー \\
general\_form.dart & 一般ユーザが入力するフォームを出力するコンポーネント \\
password\_input.dart & パスワードを入力するフォームを作成するコンポーネント \\
\hline
\multicolumn{2}{|l|}{\textbf{<配達側>}} \\
\hline
d\_notice\_change.dart & 通知画面にて配達業務と運営メールの切り替えにて使用するモジュール \\
d\_job.dart & 求人情報を表示するモジュール(求人検索での求人を選択する際に使用) \\
button\_set.dart & 任意のテキストを記載したボタンを生成するレイヤー \\
\hline
\multicolumn{2}{|l|}{\textbf{<店舗側>}} \\
\hline

\hline
\end{tabularx}
\caption{component モジュールの構成}
\end{table}


\subsection{overlay}

\subsubsection{共通機能}

\begin{table}[H]
\centering
%\caption{}
\small
\begin{tabular}{|c|l|p{0.4\textwidth}|}
\hline
モジュール名 &\multicolumn{2}{l|}{withdraw.dart}\\ \hline
モジュール概要 &\multicolumn{2}{l|}{退会を確定するかどうかを確認するオーバーレイ	}  \\ \hline
責任者 &\multicolumn{2}{l|}{梶本和希} \\ \hline
\multirow{9}{*}{構成要素} & クラス名 & Withdraw\\ \cline{2-3}
& クラス種類 & OverEntry \\ \cline{2-3}
& 処理概要 & Overlayの内容を返す \\ \cline{2-3}
& 入力 & Buildcontext context \\ \cline{2-3}
& 出力 & Material \\ \cline{2-3}
&\multicolumn{2}{l|}{他クラスとの関係} \\\cline{2-3}

&\multicolumn{2}{p{0.9\linewidth}|}{withdraw.dart の Withdraw().show(context) を呼び出すとオーバレイ表示

キャンセルが押されるとWithdraw().hide() によりcontrolle.close が呼び出されオーバレイを閉じる

退会を押すと退会処理後、Withdraw().hide() によりcontrolle.close が呼び出されオーバレイを閉じる} \\ \hline
\end{tabular}
\end{table}


%画像の追加
\begin{figure}[H]
  \centering
  \safeincludegraphics[width=0.75\textwidth]{状態遷移図/common_状態遷移図/withdraw.pdf}
  \caption{withdraw.dart の状態遷移図}
  \label{fig:withdraw.dart}
\end{figure}


\begin{table}[H]
\centering
%\caption{}
\small
\begin{tabular}{|c|l|p{0.4\textwidth}|}
\hline
モジュール名 &\multicolumn{2}{l|}{logout.dart}\\ \hline
モジュール概要 &\multicolumn{2}{l|}{ログアウトを確定するかどうかを確認するオーバーレイ	}  \\ \hline
責任者 &\multicolumn{2}{l|}{岡林磨目} \\ \hline
\multirow{9}{*}{構成要素} & クラス名 & Logout\\ \cline{2-3}
& クラス種類 & OverEntry\\ \cline{2-3}
& 処理概要 & Overlayの内容を返す \\ \cline{2-3}
& 入力 & Buildcontext context\\ \cline{2-3}
& 出力 & Material\\ \cline{2-3}
&\multicolumn{2}{l|}{他クラスとの関係} \\\cline{2-3}
&\multicolumn{2}{p{0.9\linewidth}|}{logout.dartのLogout().show(context)を呼び出すとオーバーレイ表示

キャンセルを押すと Logout().hide() によりcontrolle.close が呼び出されオーバレイを閉じる

ログアウトを押すことで login\_page.dart の LoginPage クラスを呼び出すことで、ログイン画面を表示} \\ \hline
\end{tabular}
\end{table}

%画像の追加
\begin{figure}[H]
  \centering
  \safeincludegraphics[width=0.75\textwidth]{状態遷移図/common_状態遷移図/Logout.pdf}
  \caption{logout.dart の状態遷移図}
  \label{fig:logout.dart}
\end{figure}



\begin{table}[H]
\centering
%\caption{}
\small
\begin{tabular}{|c|l|p{0.4\textwidth}|}
\hline
モジュール名 &\multicolumn{2}{l|}{rule\_screen.dart}\\ \hline
モジュール概要 &\multicolumn{2}{l|}{利用規約を表示するオーバーレイ}  \\ \hline
責任者 &\multicolumn{2}{l|}{岡林磨目} \\ \hline
\multirow{9}{*}{構成要素} & クラス名 & RuleScreen\\ \cline{2-3}
& クラス種類 & OverEntry \\ \cline{2-3}
& 処理概要 & Overlayの内容を返す \\ \cline{2-3}
& 入力 & Buildcontext context \\ \cline{2-3}
& 出力 & Material \\ \cline{2-3}
&\multicolumn{2}{l|}{他クラスとの関係} \\\cline{2-3}

&\multicolumn{2}{p{0.9\linewidth}|}{rule\_screen.dart の RuleScreen().show(context)を呼び出すとオーバーレイ表示

閉じるを押すとRuleScreen().hide()により controlle.closeが呼び出されオーバレイを閉じる} \\ \hline
\end{tabular}
\end{table}

%画像の追加
\begin{figure}[H]
  \centering
  \safeincludegraphics[width=0.75\textwidth]{状態遷移図/common_状態遷移図/RuleScreen.pdf}
  \caption{rule\_screen.dart の状態遷移図}
  \label{fig:rule_screen.dart}
\end{figure}


\begin{table}[H]
\centering
%\caption{}
\small
\begin{tabular}{|c|l|p{0.4\textwidth}|}
\hline
モジュール名 &\multicolumn{2}{l|}{cancel\_inf\_change.dart}\\ \hline
モジュール概要 &\multicolumn{2}{l|}{情報の変更をキャンセルするかを確認するオーバーレイ}  \\ \hline
責任者 &\multicolumn{2}{l|}{岡林磨目} \\ \hline
\multirow{9}{*}{構成要素} & クラス名 & CancelInfChange\\ \cline{2-3}
& クラス種類 & OverEntry \\ \cline{2-3}
& 処理概要 & Overlayの内容を返す \\ \cline{2-3}
& 入力 & Buildcontext context \\ \cline{2-3}
& 出力 & Material \\ \cline{2-3}
&\multicolumn{2}{l|}{他クラスとの関係} \\\cline{2-3}

&\multicolumn{2}{p{0.9\linewidth}|}{cancel\_inf\_change.dartの CancelInfChange().show(context) を呼び出すとオーバレイ表示

いいえを押すとCancelInfChange().hide() によりcontrolle.closeが呼び出されオーバレイを閉じる

はいを押すと変更キャンセル処理後、CancelInfChange().hide()によりcontrolle.closeが呼び出されオーバレイを閉じる} \\ \hline
\end{tabular}
\end{table}

%画像の追加
\begin{figure}[H]
  \centering
  \safeincludegraphics[width=0.75\textwidth]{状態遷移図/common_状態遷移図/CancelInfChange.pdf}
  \caption{cancel\_inf\_change.dart の状態遷移図}
  \label{fig:cancel_inf_change.dart}
\end{figure}



\begin{table}[H]
\centering
%\caption{}
\small
\begin{tabular}{|c|l|p{0.4\textwidth}|}
\hline
モジュール名 &\multicolumn{2}{l|}{confirm\_inf\_change.dart}\\ \hline
モジュール概要 &\multicolumn{2}{l|}{情報の変更を確定するかを確認するオーバーレイ}  \\ \hline
責任者 &\multicolumn{2}{l|}{岡林磨目} \\ \hline
\multirow{9}{*}{構成要素} & クラス名 & ConfirmInfChange\\ \cline{2-3}
& クラス種類 & OverEntry \\ \cline{2-3}
& 処理概要 & Overlayの内容を返す \\ \cline{2-3}
& 入力 & Buildcontext context \\ \cline{2-3}
& 出力 & Material \\ \cline{2-3}
&\multicolumn{2}{l|}{他クラスとの関係} \\\cline{2-3}

&\multicolumn{2}{p{0.9\linewidth}|}{confirm\_inf\_change.dart の ConfirmInfChange().show(context) を呼び出すとオーバレイ表示

いいえを押すと ConfirmInfChange().hide() により controlle.close が呼び出されオーバレイを閉じる

はいを押すと変更確定処理後、ConfirmInfChange().hide() により controlle.close が呼び出されオーバレイを閉じる} \\ \hline
\end{tabular}
\end{table}

%画像の追加
\begin{figure}[H]
  \centering
  \safeincludegraphics[width=0.5\textwidth]{状態遷移図/common_状態遷移図/ConfirmInfChange.pdf}
  \caption{confirm\_inf\_change.dart の状態遷移図}
  \label{fig:confirm_inf_change.dart}
\end{figure}



\subsubsection{配達側}



%画像の追加
\subsection{provider}

\begin{table}[H]
\centering
%\caption{}
\small
\begin{tabular}{|c|l|p{0.4\textwidth}|}
\hline
モジュール名 &\multicolumn{2}{l|}{change\_user\_role.dart}\\ \hline
モジュール概要 &\multicolumn{2}{l|}{アプリ全体で使用するプロバイダモジュール	}  \\ \hline
責任者 &\multicolumn{2}{l|}{岡林磨目} \\ \hline
\multirow{9}{*}{構成要素} & クラス名 & ChangeUserRole\\ \cline{2-3}
& クラス種類 & ChangeNotifier \\ \cline{2-3}
& 処理概要 & アプリ全体で使用する依頼側・配達側・店舗側の判断を行う int 型変数を定義している。
また、値が変更されると、それに対応した色(UI)へ変更が行われる。 \\ \cline{2-3}
& 入力 & なし \\ \cline{2-3}
& 出力 & なし \\ \cline{2-3}
&\multicolumn{2}{l|}{他クラスとの関係} \\\cline{2-3}

&\multicolumn{2}{p{0.9\linewidth}|}{なし} \\ \hline
\end{tabular}
\end{table}



\subsection{page}
\subsubsection{共通機能}


\begin{table}[H]
\centering
%\caption{}
\small
\begin{tabular}{|c|l|p{0.4\textwidth}|}
\hline
モジュール名 &\multicolumn{2}{l|}{change\_root\_page.dart}\\ \hline
モジュール概要 &\multicolumn{2}{l|}{アプリ全体を包括するモジュール}  \\ \hline
責任者 &\multicolumn{2}{l|}{岡林磨目} \\ \hline
\multirow{9}{*}{構成要素} & クラス名 & ChangeRootPage\\ \cline{2-3}
& クラス種類 & extends StatelessWidget \\ \cline{2-3}
& 処理概要 & アプリの大まかな区別である LoginRoute と RootRoute を包括しているクラス \\ \cline{2-3}
& 入力 & なし \\ \cline{2-3}
& 出力 & Widget \\ \cline{2-3}
&\multicolumn{2}{l|}{他クラスとの関係} \\\cline{2-3}

&\multicolumn{2}{p{0.9\linewidth}|}{app\_router.gr.dart の LoginRoute クラスを組み込む} \\ \hline
\end{tabular}
\end{table}

%画像の追加
\begin{figure}[H]
  \centering
  \safeincludegraphics[width=0.75\textwidth]{状態遷移図/common_状態遷移図/ChangeRootPage.pdf}
  \caption{change\_root\_page.dart の状態遷移図}
  \label{fig:change_root_page.dart}
\end{figure}



\begin{table}[H]
\centering
%\caption{}
\small
\begin{tabular}{|c|l|p{0.4\textwidth}|}
\hline
モジュール名 &\multicolumn{2}{l|}{new\_member.dart}\\ \hline
モジュール概要 &\multicolumn{2}{l|}{新規会員登録画面を生成するモジュール}  \\ \hline
責任者 &\multicolumn{2}{l|}{梶本和希} \\ \hline
\multirow{9}{*}{構成要素} & クラス名 & NewMember\\ \cline{2-3}
& クラス種類 & extends StatelessWidget \\ \cline{2-3}
& 処理概要 & 新規会員登録画面のWidgetを動的に変更するクラス \\ \cline{2-3}
& 入力 & なし \\ \cline{2-3}
& 出力 & Widget \\ \cline{2-3}
&\multicolumn{2}{l|}{他クラスとの関係} \\\cline{2-3}
&\multicolumn{2}{p{0.9\linewidth}|}{new\_member.dart の NewMenberState クラスを組み込む} \\ \hline

\multirow{9}{*}{構成要素} & クラス名 & NewMemberState\\ \cline{2-3}
& クラス種類 & extends StatelessWidget \\ \cline{2-3}
& 処理概要 & 新規会員登録画面を表示するクラス \\ \cline{2-3}
& 入力 & なし \\ \cline{2-3}
& 出力 & Widget \\ \cline{2-3}
&\multicolumn{2}{l|}{他クラスとの関係} \\\cline{2-3}
&\multicolumn{2}{p{0.9\linewidth}|}{rule\_screen.dart の RuleScreen クラスを呼び出すことで利用規約を表示

login\_page.dart のLoginPageクラスを呼び出しログイン画面を表示

戻るボタンの選択で前のモジュールに戻る} \\ \hline


\end{tabular}
\end{table}

%画像の追加
\begin{figure}[H]
  \centering
  \safeincludegraphics[width=0.75\textwidth]{状態遷移図/common_状態遷移図/NewMember.pdf}
  \caption{new\_member.dart の状態遷移図}
  \label{fig:new_member.dart}
\end{figure}



\begin{table}[H]
\centering
%\caption{}
\small
\begin{tabular}{|c|l|p{0.4\textwidth}|}
\hline
モジュール名 &\multicolumn{2}{l|}{mypage.dart}\\ \hline
モジュール概要 &\multicolumn{2}{l|}{マイページ画面を表示するモジュール}  \\ \hline
責任者 &\multicolumn{2}{l|}{岡林磨目} \\ \hline
\multirow{9}{*}{構成要素} & クラス名 & MyPageRouterPage \\ \cline{2-3}
& クラス種類 & extends AppRouter \\ \cline{2-3}
& 処理概要 & MyPage にアクセスするための Router \\ \cline{2-3}
& 入力 & なし \\ \cline{2-3}
& 出力 & Route \\ \cline{2-3}
&\multicolumn{2}{l|}{他クラスとの関係} \\\cline{2-3}
&\multicolumn{2}{p{0.9\linewidth}|}{なし} \\ \hline

\multirow{9}{*}{構成要素} & クラス名 & MyPage\\ \cline{2-3}
& クラス種類 & extends StatelessWidget \\ \cline{2-3}
& 処理概要 & マイページ画面を動的に変更するクラス \\ \cline{2-3}
& 入力 & なし \\ \cline{2-3}
& 出力 & Widget \\ \cline{2-3}
&\multicolumn{2}{l|}{他クラスとの関係} \\\cline{2-3}
&\multicolumn{2}{p{0.9\linewidth}|}{MyPage.dart の MyPageState クラスを組み込む} \\ \hline


\multirow{9}{*}{構成要素} & クラス名 & MyPageState\\ \cline{2-3}
& クラス種類 & extends StatelessWidget \\ \cline{2-3}
& 処理概要 & マイページ画面の状態を返すクラス \\ \cline{2-3}
& 入力 & なし \\ \cline{2-3}
& 出力 & State \\ \cline{2-3}
&\multicolumn{2}{l|}{他クラスとの関係} \\\cline{2-3}
&\multicolumn{2}{p{0.9\linewidth}|}{title\_appbar.dart の TitleAppBar クラスを組み込む

mypage.dart の ScrollMyPageDetail クラスを組み込む} \\ \hline
\end{tabular}
\end{table}

%画像の追加
\begin{figure}[H]
  \centering
  \safeincludegraphics[width=0.75\textwidth]{状態遷移図/common_状態遷移図/MyPage.pdf}
  \caption{mypage.dart の状態遷移図}
  \label{fig:mypage.dart}
\end{figure}



\begin{table}[H]
\centering
%\caption{}
\small
\begin{tabular}{|c|l|p{0.4\textwidth}|}
\hline
モジュール名 &\multicolumn{2}{l|}{login\_page.dart}\\ \hline
モジュール概要 &\multicolumn{2}{l|}{ログイン画面を生成するモジュール}  \\ \hline
責任者 &\multicolumn{2}{l|}{梶本和希} \\ \hline
\multirow{9}{*}{構成要素} & クラス名 & LoginPage\\ \cline{2-3}
& クラス種類 & extends StatelessWidget \\ \cline{2-3}
& 処理概要 & ログイン画面のWidgetを動的に変更するクラス \\ \cline{2-3}
& 入力 & なし \\ \cline{2-3}
& 出力 & Widget \\ \cline{2-3}
&\multicolumn{2}{l|}{他クラスとの関係} \\\cline{2-3}
&\multicolumn{2}{p{0.9\linewidth}|}{login\_page.dartのLoginPageStateクラスを組み込む} \\ \hline

\multirow{9}{*}{構成要素} & クラス名 & LoginPageState \\ \cline{2-3}
& クラス種類 & extends StatelessWidget \\ \cline{2-3}
& 処理概要 & ログイン画面の状態を返すクラス \\ \cline{2-3}
& 入力 & なし \\ \cline{2-3}
& 出力 & State \\ \cline{2-3}
&\multicolumn{2}{l|}{他クラスとの関係} \\\cline{2-3}
&\multicolumn{2}{p{0.9\linewidth}|}{new\_member.dartのNewMemberクラスを呼び出すことで、新規会員登録画面を表示

login\_page.dartのLoginAppBarクラスを組み込む} \\ \hline


\multirow{9}{*}{構成要素} & クラス名 & LoginAppBar\\ \cline{2-3}
& クラス種類 & extends StatelessWidget,imprementsPrefferdSizeWidget \\ \cline{2-3}
& 処理概要 & ログイン画面用に作成を行ったAppBarを生成するコンポーネント \\ \cline{2-3}
& 入力 & ChangeGeneralCorporationstore \\ \cline{2-3}
& 出力 & Widget \\ \cline{2-3}
&\multicolumn{2}{l|}{他クラスとの関係} \\\cline{2-3}
&\multicolumn{2}{p{0.9\linewidth}|}{なし} \\ \hline

\end{tabular}
\end{table}

%画像の追加(ファイルが存在する場合のみ表示)
\IfFileExists{状態遷移図/common_状態遷移図/login.pdf}{%
  \begin{figure}[H]
    \centering
    \safeincludegraphics[width=0.75\textwidth]{状態遷移図/common_状態遷移図/login.pdf}
    \caption{login\_page.dart の状態遷移図}
    \label{fig:login_page.dart}
  \end{figure}
}{%
  % 図ファイル未配置時はプレースホルダのみ表示
  \begin{figure}[H]
    \centering
    \fbox{\begin{minipage}{0.7\textwidth}\centering
      状態遷移図/common\_状態遷移図/LoginPage.pdf が存在しません。
    \end{minipage}}
    \caption{login\_page.dart の状態遷移図(ファイル未配置)}
    \label{fig:login_page.dart}
  \end{figure}
}



\begin{table}[H]
\centering
%\caption{}
\small
\begin{tabular}{|c|l|p{0.4\textwidth}|}
\hline
モジュール名 &\multicolumn{2}{l|}{logout\_withdraw\_page.dart}\\ \hline
モジュール概要 &\multicolumn{2}{l|}{ログアウト・退会画面を生成するモジュール}  \\ \hline
責任者 &\multicolumn{2}{l|}{梶本和希} \\ \hline
\multirow{9}{*}{構成要素} & クラス名 & LogoutWithdraw\\ \cline{2-3}
& クラス種類 & extends StatelessWidget \\ \cline{2-3}
& 処理概要 &  ログアウト・退会画面を生成するクラス\\ \cline{2-3}
& 入力 & なし \\ \cline{2-3}
& 出力 & Widget \\ \cline{2-3}
&\multicolumn{2}{l|}{他クラスとの関係} \\\cline{2-3}

&\multicolumn{2}{p{0.9\linewidth}|}{withdraw.dartのWithdrawクラスの呼び出すことで退会オーバーレイを表示
logout.dartのLogoutクラスの呼び出すことでログアウトオーバーレイを表示
戻るボタンの選択で前のモジュールに戻る} \\ \hline
\end{tabular}
\end{table}

%画像の追加
\begin{figure}[H]
  \centering
  \safeincludegraphics[width=0.75\textwidth]{状態遷移図/common_状態遷移図/LogoutWithdraw.pdf}
  \caption{logout\_withdraw\_page.dart の状態遷移図}
  \label{fig:logout_withdraw_page.dart}
\end{figure}




\subsubsection{依頼側}

% 店舗検索・マップ
\begin{table}[H]
\centering
\small
\begin{tabular}{|c|l|p{0.4\textwidth}|}
\hline
モジュール名 &\multicolumn{2}{l|}{c\_map\_search.dart}\\ \hline
モジュール概要 &\multicolumn{2}{l|}{マップでの店舗検索を行う画面を表示するモジュール}  \\ \hline
責任者 &\multicolumn{2}{l|}{工藤 愛} \\ \hline
\multirow{9}{*}{構成要素} & クラス名 & MapSearch\\ \cline{2-3}
& クラス種類 & extends StatelessWidget \\ \cline{2-3}
& 処理概要 & マップと店舗検索バーを表示する \\ \cline{2-3}
& 入力 & なし \\ \cline{2-3}
& 出力 & Widget \\ \cline{2-3}
&\multicolumn{2}{l|}{他クラスとの関係} \\\cline{2-3}
&\multicolumn{2}{p{0.9\linewidth}|}{GoogleMap Widgetを表示し、検索結果に応じてピンを配置する} \\ \hline
\end{tabular}
\end{table}

%画像の追加
\begin{figure}[H]
  \centering
  \safeincludegraphics[width=0.75\textwidth]{状態遷移図/c_map_search.dart.drawio.pdf}
  \caption{c\_map\_search.dart の状態遷移図}
  \label{fig:c_map_search.dart}
\end{figure}

% 商品一覧
\begin{table}[H]
\centering
\small
\begin{tabular}{|c|l|p{0.4\textwidth}|}
\hline
モジュール名 &\multicolumn{2}{l|}{c\_product\_list.dart}\\ \hline
モジュール概要 &\multicolumn{2}{l|}{商品一覧画面を表示するモジュール}  \\ \hline
責任者 &\multicolumn{2}{l|}{工藤 愛} \\ \hline
\multirow{9}{*}{構成要素} & クラス名 & ProductList\\ \cline{2-3}
& クラス種類 & extends StatelessWidget \\ \cline{2-3}
& 処理概要 & 選択された店舗の商品一覧を表示する \\ \cline{2-3}
& 入力 & int storeId \\ \cline{2-3}
& 出力 & Widget \\ \cline{2-3}
&\multicolumn{2}{l|}{他クラスとの関係} \\\cline{2-3}
&\multicolumn{2}{p{0.9\linewidth}|}{APIから商品情報を取得し、Card Widget等でリスト表示する。商品選択でカートに追加または詳細表示を行う} \\ \hline
\end{tabular}
\end{table}

%画像の追加
\begin{figure}[H]
  \centering
  \safeincludegraphics[width=0.75\textwidth]{状態遷移図/c_product_list.dart.drawio.pdf}
  \caption{c\_product\_list.dart の状態遷移図}
  \label{fig:c_product_list.dart}
\end{figure}

% カート
\begin{table}[H]
\centering
\small
\begin{tabular}{|c|l|p{0.4\textwidth}|}
\hline
モジュール名 &\multicolumn{2}{l|}{c\_cart.dart}\\ \hline
モジュール概要 &\multicolumn{2}{l|}{カート情報を表示するモジュール}  \\ \hline
責任者 &\multicolumn{2}{l|}{工藤 愛} \\ \hline
\multirow{9}{*}{構成要素} & クラス名 & Cart\\ \cline{2-3}
& クラス種類 & extends StatelessWidget \\ \cline{2-3}
& 処理概要 & 現在カートに入っている商品リストと合計金額を表示する \\ \cline{2-3}
& 入力 & なし \\ \cline{2-3}
& 出力 & Widget \\ \cline{2-3}
&\multicolumn{2}{l|}{他クラスとの関係} \\\cline{2-3}
&\multicolumn{2}{p{0.9\linewidth}|}{c\_order\_check.dartへ遷移し注文確認を行う} \\ \hline
\end{tabular}
\end{table}

%画像の追加
\begin{figure}[H]
  \centering
  \safeincludegraphics[width=0.75\textwidth]{状態遷移図/c_cart.dart.drawio.pdf}
  \caption{c\_cart.dart の状態遷移図}
  \label{fig:c_cart.dart}
\end{figure}

% 注文確認・確定
\begin{table}[H]
\centering
\small
\begin{tabular}{|c|l|p{0.4\textwidth}|}
\hline
モジュール名 &\multicolumn{2}{l|}{c\_order\_confirm.dart}\\ \hline
モジュール概要 &\multicolumn{2}{l|}{注文確定画面を表示するモジュール}  \\ \hline
責任者 &\multicolumn{2}{l|}{工藤 愛} \\ \hline
\multirow{9}{*}{構成要素} & クラス名 & OrderConfirm\\ \cline{2-3}
& クラス種類 & extends StatelessWidget \\ \cline{2-3}
& 処理概要 & 注文内容の最終確認と確定処理を行う \\ \cline{2-3}
& 入力 & OrderData order \\ \cline{2-3}
& 出力 & Widget \\ \cline{2-3}
&\multicolumn{2}{l|}{他クラスとの関係} \\\cline{2-3}
&\multicolumn{2}{p{0.9\linewidth}|}{APIへ注文データを送信し、成功時に完了画面(finish\_screen.dart)を表示する} \\ \hline
\end{tabular}
\end{table}

%画像の追加
\begin{figure}[H]
  \centering
  \safeincludegraphics[width=0.75\textwidth]{状態遷移図/c_order_confirm.dart.drawio.pdf}
  \caption{c\_order\_confirm.dart の状態遷移図}
  \label{fig:c_order_confirm.dart}
\end{figure}

% 支払い明細
\begin{table}[H]
\centering
\small
\begin{tabular}{|c|l|p{0.4\textwidth}|}
\hline
モジュール名 &\multicolumn{2}{l|}{c\_payment.dart}\\ \hline
モジュール概要 &\multicolumn{2}{l|}{支払い明細画面を表示するモジュール}  \\ \hline
責任者 &\multicolumn{2}{l|}{工藤 愛} \\ \hline
\multirow{9}{*}{構成要素} & クラス名 & PaymentDetail\\ \cline{2-3}
& クラス種類 & extends StatelessWidget \\ \cline{2-3}
& 処理概要 & 注文の詳細な支払い内訳を表示する \\ \cline{2-3}
& 入力 & int orderId \\ \cline{2-3}
& 出力 & Widget \\ \cline{2-3}
&\multicolumn{2}{l|}{他クラスとの関係} \\\cline{2-3}
&\multicolumn{2}{p{0.9\linewidth}|}{なし} \\ \hline
\end{tabular}
\end{table}

%画像の追加
\begin{figure}[H]
  \centering
  \safeincludegraphics[width=0.75\textwidth]{状態遷移図/c_payment.dart.drawio.pdf}
  \caption{c\_payment.dart の状態遷移図}
  \label{fig:c_payment.dart}
\end{figure}

% 商品到着報告
\begin{table}[H]
\centering
\small
\begin{tabular}{|c|l|p{0.4\textwidth}|}
\hline
モジュール名 &\multicolumn{2}{l|}{c\_order\_arrival\_report.dart}\\ \hline
モジュール概要 &\multicolumn{2}{l|}{商品到着報告画面を表示するモジュール}  \\ \hline
責任者 &\multicolumn{2}{l|}{工藤 愛} \\ \hline
\multirow{9}{*}{構成要素} & クラス名 & ArrivalReport\\ \cline{2-3}
& クラス種類 & extends StatelessWidget \\ \cline{2-3}
& 処理概要 & 商品受け取りを確認し、完了報告を行う \\ \cline{2-3}
& 入力 & int orderId \\ \cline{2-3}
& 出力 & Widget \\ \cline{2-3}
&\multicolumn{2}{l|}{他クラスとの関係} \\\cline{2-3}
&\multicolumn{2}{p{0.9\linewidth}|}{報告ボタン押下でステータス更新APIを呼び出す} \\ \hline
\end{tabular}
\end{table}

%画像の追加
\begin{figure}[H]
  \centering
  \safeincludegraphics[width=0.75\textwidth]{状態遷移図/c_order_arrival_report.dart.drawio.pdf}
  \caption{c\_order\_arrival\_report.dart の状態遷移図}
  \label{fig:c_order_arrival_report.dart}
\end{figure}

% 注文履歴
\begin{table}[H]
\centering
\small
\begin{tabular}{|c|l|p{0.4\textwidth}|}
\hline
モジュール名 &\multicolumn{2}{l|}{c\_order\_history.dart}\\ \hline
モジュール概要 &\multicolumn{2}{l|}{注文履歴一覧画面を表示するモジュール}  \\ \hline
責任者 &\multicolumn{2}{l|}{工藤 愛} \\ \hline
\multirow{9}{*}{構成要素} & クラス名 & OrderHistory\\ \cline{2-3}
& クラス種類 & extends StatelessWidget \\ \cline{2-3}
& 処理概要 & 過去の注文履歴をリスト形式で表示する \\ \cline{2-3}
& 入力 & なし \\ \cline{2-3}
& 出力 & Widget \\ \cline{2-3}
&\multicolumn{2}{l|}{他クラスとの関係} \\\cline{2-3}
&\multicolumn{2}{p{0.9\linewidth}|}{リスト選択でc\_order\_history\_detail.dartへ遷移する} \\ \hline
\end{tabular}
\end{table}

%画像の追加
\begin{figure}[H]
  \centering
  \safeincludegraphics[width=0.75\textwidth]{状態遷移図/c_order_history.dart.drawio.pdf}
  \caption{c\_order\_history.dart の状態遷移図}
  \label{fig:c_order_history.dart}
\end{figure}

% 住所確認・編集
\begin{table}[H]
\centering
\small
\begin{tabular}{|c|l|p{0.4\textwidth}|}
\hline
モジュール名 &\multicolumn{2}{l|}{c\_address.dart}\\ \hline
モジュール概要 &\multicolumn{2}{l|}{配達先住所確認画面を表示するモジュール}  \\ \hline
責任者 &\multicolumn{2}{l|}{工藤 愛} \\ \hline
\multirow{9}{*}{構成要素} & クラス名 & AddressInfo\\ \cline{2-3}
& クラス種類 & extends StatelessWidget \\ \cline{2-3}
& 処理概要 & 登録されている配達先住所を表示する \\ \cline{2-3}
& 入力 & なし \\ \cline{2-3}
& 出力 & Widget \\ \cline{2-3}
&\multicolumn{2}{l|}{他クラスとの関係} \\\cline{2-3}
&\multicolumn{2}{p{0.9\linewidth}|}{編集ボタンでc\_address\_change.dart、追加ボタンでc\_address\_add.dartへ遷移する} \\ \hline
\end{tabular}
\end{table}

%画像の追加
\begin{figure}[H]
  \centering
  \safeincludegraphics[width=0.75\textwidth]{状態遷移図/c_address.dart.drawio.pdf}
  \caption{c\_address.dart の状態遷移図}
  \label{fig:c_address.dart}
\end{figure}

% 口座情報
\begin{table}[H]
\centering
\small
\begin{tabular}{|c|l|p{0.4\textwidth}|}
\hline
モジュール名 &\multicolumn{2}{l|}{c\_banking\_info.dart}\\ \hline
モジュール概要 &\multicolumn{2}{l|}{口座情報管理画面を表示するモジュール}  \\ \hline
責任者 &\multicolumn{2}{l|}{工藤 愛} \\ \hline
\multirow{9}{*}{構成要素} & クラス名 & BankingInfo\\ \cline{2-3}
& クラス種類 & extends StatelessWidget \\ \cline{2-3}
& 処理概要 & 登録口座情報を表示する \\ \cline{2-3}
& 入力 & なし \\ \cline{2-3}
& 出力 & Widget \\ \cline{2-3}
&\multicolumn{2}{l|}{他クラスとの関係} \\\cline{2-3}
&\multicolumn{2}{p{0.9\linewidth}|}{変更ボタンでc\_banking\_info\_change.dartへ遷移する} \\ \hline
\end{tabular}
\end{table}

%画像の追加
\begin{figure}[H]
  \centering
  \safeincludegraphics[width=0.75\textwidth]{状態遷移図/c_banking_info.dart.drawio.pdf}
  \caption{c\_banking\_info.dart の状態遷移図}
  \label{fig:c_banking_info.dart}
\end{figure}



\subsubsection{配達側}


\begin{table}[h]
\centering
%\caption{}
\small
\begin{tabular}{|c|l|p{0.4\textwidth}|}
\hline
モジュール名 &\multicolumn{2}{l|}{d\_notice.dart}\\ \hline
モジュール概要 &\multicolumn{2}{l|}{通知画面を表示するモジュール}  \\ \hline
責任者 &\multicolumn{2}{l|}{金澤優} \\ \hline
\multirow{9}{*}{構成要素} & クラス名 & DRootPage\\ \cline{2-3}
& クラス種類 & extendsStatelessWidget \\ \cline{2-3}
& 処理概要 & ログイン以外のアプリの包括を行い、ボトムアップバーを作成するクラス
 \\ \cline{2-3}
& 入力 & なし \\ \cline{2-3}
& 出力 & Widget \\ \cline{2-3}
&\multicolumn{2}{l|}{他クラスとの関係} \\\cline{2-3}

&\multicolumn{2}{p{0.9\linewidth}|}{rule\_screen.dart の RuleScreen().show(context)を呼び出すとオーバーレイ表示

閉じるを押すとRuleScreen().hide()により controlle.closeが呼び出されオーバレイを閉じる} \\ \hline
\end{tabular}
\end{table}

\begin{figure}[H]
  \centering
  \includegraphics[width=0.8\textwidth]{./状態遷移図/delivery_状態遷移図/d_root_page.pdf}
  \caption{d\_root\_page.dart の状態遷移図}
  \label{fig:d_root_page.dart}
\end{figure}

\newpage

%d_notice_detail.dart
\begin{table}[H]
\centering
%\caption{}
\small
\begin{tabular}{|c|l|p{0.4\textwidth}|}
\hline
モジュール名 &\multicolumn{2}{l|}{d\_notice.dart}\\ \hline
モジュール概要 &\multicolumn{2}{l|}{通知画面を表示するモジュール}  \\ \hline
責任者 &\multicolumn{2}{l|}{金澤優} \\ \hline
\multirow{9}{*}{構成要素} & クラス名 & DNotice
\\ \cline{2-3}
& クラス種類 & extends StatelessWidget \\ \cline{2-3}
& 処理概要 & 通知画面のWidgetを動的に変更するクラス \\ \cline{2-3}
& 入力 & なし \\ \cline{2-3}
& 出力 & Widget \\ \cline{2-3}
&\multicolumn{2}{l|}{他クラスとの関係} \\\cline{2-3}
&\multicolumn{2}{p{0.9\linewidth}|}{d\_notice.dartにNoticeStateクラスを組み込む} \\ \hline

\multirow{9}{*}{構成要素} & クラス名 & DNoticeState \\ \cline{2-3}
& クラス種類 & extends StatelessWidget \\ \cline{2-3}
& 処理概要 & 通知画面の状態を返す \\ \cline{2-3}
& 入力 & なし \\ \cline{2-3}
& 出力 & State \\ \cline{2-3}
&\multicolumn{2}{l|}{他クラスとの関係} \\\cline{2-3}
&\multicolumn{2}{p{0.9\linewidth}|}{d\_notice.dartにDNoticeStateクラスを組み込む} \\ \hline

\end{tabular}
\end{table}

\begin{figure}[H]
  \centering
  \includegraphics[width=0.8\textwidth]{./状態遷移図/delivery_状態遷移図/d_notice.pdf}
  \caption{d\_notice.dart の状態遷移図}
  \label{fig:d_notice}
\end{figure}


\newpage

%d_notice_detail.dart
\begin{table}[h]
\centering
%\caption{}
\small
\begin{tabular}{|c|l|p{0.4\textwidth}|}
\hline
モジュール名 &\multicolumn{2}{l|}{d\_notice\_detail.dart}\\ \hline
モジュール概要 &\multicolumn{2}{l|}{通知の詳細画面を表示するモジュール} \\ \hline
責任者 &\multicolumn{2}{l|}{金澤優} \\ \hline
\multirow{9}{*}{構成要素} & クラス名 & DNoticeDetail\\ \cline{2-3}
& クラス種類 & extendsStatelessWidget \\ \cline{2-3}
& 処理概要 & 任意の通知詳細画面を生成するクラス
 \\ \cline{2-3}
& 入力 & List<List<Map>>titles, intisevent, intindex,Stringde tail \\ \cline{2-3}
& 出力 & Widget \\ \cline{2-3}
&\multicolumn{2}{l|}{他クラスとの関係} 
\\\cline{2-3}

&\multicolumn{2}{p{0.9\linewidth}|}{withdraw.dart の Withdraw().show(context) を呼び出すとオーバレイ表示

キャンセルが押されるとWithdraw().hide() によりcontrolle.close が呼び出されオーバレイを閉じる

退会を押すと退会処理後、Withdraw().hide() によりcontrolle.close が呼び出されオーバレイを閉じる} \\ \hline
\end{tabular}
\end{table}

\begin{figure}[H]
  \centering
  \includegraphics[width=0.8\textwidth]{./状態遷移図/delivery_状態遷移図/d_notice_detail.pdf}
  \caption{d\_notice\_detail.dart の状態遷移図}
  \label{fig:d_notice_detail}
\end{figure}

\newpage

%d_home.dart
\begin{table}[H]
\centering
%\caption{}
\small
\begin{tabular}{|c|l|p{0.4\textwidth}|}
\hline
モジュール名 &\multicolumn{2}{l|}{d\_home.dart}\\ \hline
モジュール概要 &\multicolumn{2}{l|}{配達員側のホーム画面を生成するモジュール}  \\ \hline
責任者 &\multicolumn{2}{l|}{金澤優} \\ \hline
\multirow{9}{*}{構成要素} & クラス名 & DHomeRoutePage
\\ \cline{2-3}
& クラス種類 & extends AppRouter\\ \cline{2-3}
& 処理概要 & ホーム(Home)をアクセスするためのRouter \\ \cline{2-3}
& 入力 & なし \\ \cline{2-3}
& 出力 & Router \\ \cline{2-3}
&\multicolumn{2}{l|}{他クラスとの関係} \\\cline{2-3}
&\multicolumn{2}{p{0.9\linewidth}|}{なし} \\ \hline

\multirow{9}{*}{構成要素} & クラス名 & DHome \\ \cline{2-3}
& クラス種類 & extends StatelessWidget \\ \cline{2-3}
& 処理概要 & ホーム画面のWidgetを動的に変更するクラス \\ \cline{2-3}
& 入力 & なし \\ \cline{2-3}
& 出力 & State \\ \cline{2-3}
&\multicolumn{2}{l|}{他クラスとの関係} \\\cline{2-3}
&\multicolumn{2}{p{0.9\linewidth}|}{d\_home.dartにDHomeStateクラスを組み込む} \\ \hline

\multirow{9}{*}{構成要素} & クラス名 & DHomeState \\ \cline{2-3}
& クラス種類 & extends StatelessWidget \\ \cline{2-3}
& 処理概要 & ホーム画面の状態を返すクラス \\ \cline{2-3}
& 入力 & なし \\ \cline{2-3}
& 出力 & State \\ \cline{2-3}
&\multicolumn{2}{l|}{他クラスとの関係} \\\cline{2-3}
&\multicolumn{2}{p{0.9\linewidth}|}{なし} \\ \hline

\end{tabular}
\end{table}

\begin{figure}[H]
  \centering
  \includegraphics[width=0.8\textwidth]{./状態遷移図/delivery_状態遷移図/d_notice.pdf}
  \caption{d\_home.dart の状態遷移図}
  \label{fig:d_notice}
\end{figure}

\newpage

%d_home.dart
\begin{table}[H]
\centering
%\caption{}
\small
\begin{tabular}{|c|l|p{0.4\textwidth}|}
\hline
モジュール名 &\multicolumn{2}{l|}{d\_resume.dart}\\ \hline
モジュール概要 &\multicolumn{2}{l|}{履歴書情報を表示するモジュール}  \\ \hline
責任者 &\multicolumn{2}{l|}{金澤優} \\ \hline
\multirow{9}{*}{構成要素} & クラス名 & DResumeChange
\\ \cline{2-3}
& クラス種類 & extends StatelessWidget\\ \cline{2-3}
& 処理概要 & 履歴書画面のWidgetを動的に変更するクラス \\ \cline{2-3}
& 入力 & なし \\ \cline{2-3}
& 出力 & Router \\ \cline{2-3}
&\multicolumn{2}{l|}{他クラスとの関係} \\\cline{2-3}
&\multicolumn{2}{p{0.9\linewidth}|}{d\_resume.dartにDResumeChangeStateクラスを組み込む} \\ \hline

\multirow{9}{*}{構成要素} & クラス名 & DHome \\ \cline{2-3}
& クラス種類 & extends StatelessWidget \\ \cline{2-3}
& 処理概要 & ホーム画面のWidgetを動的に変更するクラス \\ \cline{2-3}
& 入力 & なし \\ \cline{2-3}
& 出力 & State \\ \cline{2-3}
&\multicolumn{2}{l|}{他クラスとの関係} \\\cline{2-3}
&\multicolumn{2}{p{0.9\linewidth}|}{d\_home.dartにDHomeStateクラスを組み込む} \\ \hline

\multirow{9}{*}{構成要素} & クラス名 & DHomeState \\ \cline{2-3}
& クラス種類 & extends StatelessWidget \\ \cline{2-3}
& 処理概要 & ホーム画面の状態を返すクラス \\ \cline{2-3}
& 入力 & なし \\ \cline{2-3}
& 出力 & State \\ \cline{2-3}
&\multicolumn{2}{l|}{他クラスとの関係} \\\cline{2-3}
&\multicolumn{2}{p{0.9\linewidth}|}{なし} \\ \hline

\end{tabular}
\end{table}

\begin{figure}[H]
  \centering
  \includegraphics[width=0.8\textwidth]{./状態遷移図/delivery_状態遷移図/d_notice.pdf}
  \caption{d\_home.dart の状態遷移図}
  \label{fig:d_notice}
\end{figure}

\newpage

%d_home.dart
\begin{table}[H]
\centering
%\caption{}
\small
\begin{tabular}{|c|l|p{0.4\textwidth}|}
\hline
モジュール名 &\multicolumn{2}{l|}{d\_member\_information.dart}\\ \hline
モジュール概要 &\multicolumn{2}{l|}{配達員の会員情報を確認する画面を表示するモジュール}  \\ \hline
責任者 &\multicolumn{2}{l|}{金澤優} \\ \hline
\multirow{9}{*}{構成要素} & クラス名 & DMemberInformation
\\ \cline{2-3}
& クラス種類 & extends StatelessWidget\\ \cline{2-3}
& 処理概要 & 配達員の会員情報をWidgetを動的に変更するクラス \\ \cline{2-3}
& 入力 & なし \\ \cline{2-3}
& 出力 & Widget \\ \cline{2-3}
&\multicolumn{2}{l|}{他クラスとの関係} \\\cline{2-3}
&\multicolumn{2}{p{0.9\linewidth}|}{d\_member\_information.dartのDMemberInformationクラスを組み込む} \\ \hline

\multirow{9}{*}{構成要素} & クラス名 & DMemberInformationState \\ \cline{2-3}
& クラス種類 & extends StatelessWidget \\ \cline{2-3}
& 処理概要 & 配達員の会員情報の状態を返す \\ \cline{2-3}
& 入力 & なし \\ \cline{2-3}
& 出力 & State \\ \cline{2-3}
&\multicolumn{2}{l|}{他クラスとの関係} \\\cline{2-3}
&\multicolumn{2}{p{0.9\linewidth}|}{d\_home.dartにDHomeStateクラスを組み込む} \\ \hline

\multirow{9}{*}{構成要素} & クラス名 & DHomeState \\ \cline{2-3}
& クラス種類 & extends StatelessWidget \\ \cline{2-3}
& 処理概要 & ホーム画面の状態を返すクラス \\ \cline{2-3}
& 入力 & なし \\ \cline{2-3}
& 出力 & State \\ \cline{2-3}
&\multicolumn{2}{l|}{他クラスとの関係} \\\cline{2-3}
&\multicolumn{2}{p{0.9\linewidth}|}{change general corporation.dart の ChangeGeneralCorporation プロバイダを組み込む
title appbar.dart の TitleAppBar クラスを組み込む
general mem inf change.dart の GeneralMemInfChange クラスを呼び出し一般会員情報変更
画面を表示
pass change.dart の PassChange クラスを呼び出しパスワード変更画面を表示
戻るボタンの選択で前のモジュールに戻る} \\ \hline

\end{tabular}
\end{table}

\begin{figure}[H]
  \centering
  \includegraphics[width=0.8\textwidth]{./状態遷移図/delivery_状態遷移図/d_member_information.pdf}
  \caption{d\_member\_information.dart の状態遷移図}
  \label{fig:d_notice}
\end{figure}

\newpage

%d_home.dart
\begin{table}[H]
\centering
%\caption{}
\small
\begin{tabular}{|c|l|p{0.4\textwidth}|}
\hline
モジュール名 &\multicolumn{2}{l|}{d\_mem\_inf_change.dart}\\ \hline
モジュール概要 &\multicolumn{2}{l|}{配達員の会員情報変更画面を表示するモジュール}  \\ \hline
責任者 &\multicolumn{2}{l|}{金澤優} \\ \hline
\multirow{9}{*}{構成要素} & クラス名 & DMemberInformationChange
\\ \cline{2-3}
& クラス種類 & extends StatelessWidget\\ \cline{2-3}
& 処理概要 & 配達員の会員情報変更画面をWidgetを動的に変更するクラス \\ \cline{2-3}
& 入力 & なし \\ \cline{2-3}
& 出力 & Widget \\ \cline{2-3}
&\multicolumn{2}{l|}{他クラスとの関係} \\\cline{2-3}
&\multicolumn{2}{p{0.9\linewidth}|}{d_mem_inf_change.dartのDMemberInformationChangeStateクラスを組み込む} \\ \hline

\multirow{9}{*}{構成要素} & クラス名 & DMemberInformationChangeState \\ \cline{2-3}
& クラス種類 & extends StatelessWidget \\ \cline{2-3}
& 処理概要 & 配達員の会員情報変更画面の状態を返す\\ \cline{2-3}
& 入力 & なし \\ \cline{2-3}
& 出力 & State \\ \cline{2-3}
&\multicolumn{2}{l|}{他クラスとの関係} \\\cline{2-3}
&\multicolumn{2}{p{0.9\linewidth}|}{change general corporation.dart の ChangeGeneralCorporation プロバイダを組み込む

title appbar.dart の TitleAppBar クラスを組み込む

general mem inf conf.dart のGeneralMemInfConfクラスを呼び出し一般会員情報確認画面を
表示

戻るボタンの選択で前のモジュールに戻る} \\ \hline
\end{tabular}
\end{table}

\begin{figure}[H]
  \centering
  \includegraphics[width=0.8\textwidth]{./状態遷移図/delivery_状態遷移図/d_mem_inf_change.pdf}
  \caption{d\_mem\_inf\_change.dart の状態遷移図}
  \label{fig:d_mem_inf_change}
\end{figure}

\newpage

%d_home.dart
\begin{table}[H]
\centering
%\caption{}
\small
\begin{tabular}{|c|l|p{0.4\textwidth}|}
\hline
モジュール名 &\multicolumn{2}{l|}{d\_banking\_information.dart}\\ \hline
モジュール概要 &\multicolumn{2}{l|}{口座情報を示した画面を表示するモジュール}  \\ \hline
責任者 &\multicolumn{2}{l|}{金澤優} \\ \hline
\multirow{9}{*}{構成要素} & クラス名 & DBankingInformation
\\ \cline{2-3}
& クラス種類 & extends StatelessWidget\\ \cline{2-3}
& 処理概要 & 口座情報確認画面を動的に変更するクラス \\ \cline{2-3}
& 入力 & なし \\ \cline{2-3}
& 出力 & Widget \\ \cline{2-3}
&\multicolumn{2}{l|}{他クラスとの関係} \\\cline{2-3}
&\multicolumn{2}{p{0.9\linewidth}|}{d_mem_inf_change.dartのDBankingInformationStateクラスを組み込む} \\ \hline

\multirow{9}{*}{構成要素} & クラス名 & DBankingInformationState \\ \cline{2-3}
& クラス種類 & extends StatelessWidget \\ \cline{2-3}
& 処理概要 & 配達員の口座情報変更画面の状態を返す\\ \cline{2-3}
& 入力 & なし \\ \cline{2-3}
& 出力 & State \\ \cline{2-3}
&\multicolumn{2}{l|}{他クラスとの関係} \\\cline{2-3}
&\multicolumn{2}{p{0.9\linewidth}|}{change general corporation.dart の ChangeGeneralCorporation プロバイダを組み込む

title appbar.dart の TitleAppBar クラスを組み込む

general mem inf conf.dart のGeneralMemInfConfクラスを呼び出し一般会員情報確認画面を表示

戻るボタンの選択で前のモジュールに戻る} \\ \hline
\end{tabular}
\end{table}

\begin{figure}[H]
  \centering
  \includegraphics[width=0.8\textwidth]{./状態遷移図/delivery_状態遷移図/d_mem_inf_change.pdf}
  \caption{d\_banking\_information.dart の状態遷移図}
  \label{fig:d_mem_inf_change}
\end{figure}



\newpage

%d_home.dart
\begin{table}[H]
\centering
%\caption{}
\small
\begin{tabular}{|c|l|p{0.4\textwidth}|}
\hline
モジュール名 &\multicolumn{2}{l|}{d\_bank\_inf\_change.dart}\\ \hline
モジュール概要 &\multicolumn{2}{l|}{口座情報変更画面を表示するモジュール}  \\ \hline
責任者 &\multicolumn{2}{l|}{金澤優} \\ \hline
\multirow{9}{*}{構成要素} & クラス名 & DBankingInformationChange
\\ \cline{2-3}
& クラス種類 & extends StatelessWidget\\ \cline{2-3}
& 処理概要 & 口座情報変更画面を動的に変更するクラス \\ \cline{2-3}
& 入力 & なし \\ \cline{2-3}
& 出力 & Widget \\ \cline{2-3}
&\multicolumn{2}{l|}{他クラスとの関係} \\\cline{2-3}
&\multicolumn{2}{p{0.9\linewidth}|}{d_mem_inf_change.dartのDBankingInformationChangeStateクラスを組み込む} \\ \hline

\multirow{9}{*}{構成要素} & クラス名 & DBankingInformationChangeState \\ \cline{2-3}
& クラス種類 & extends StatelessWidget \\ \cline{2-3}
& 処理概要 & 配達員の口座情報変更画面の状態を返す\\ \cline{2-3}
& 入力 & なし \\ \cline{2-3}
& 出力 & State \\ \cline{2-3}
&\multicolumn{2}{l|}{他クラスとの関係} \\\cline{2-3}
&\multicolumn{2}{p{0.9\linewidth}|}{change general corporation.dart の ChangeGeneralCorporation プロバイダを組み込む

title appbar.dart の TitleAppBar クラスを組み込む

general mem inf conf.dart のGeneralMemInfConfクラスを呼び出し一般会員情報確認画面を表示

戻るボタンの選択で前のモジュールに戻る} \\ \hline
\end{tabular}
\end{table}

\begin{figure}[H]
  \centering
  \includegraphics[width=0.8\textwidth]{./状態遷移図/delivery_状態遷移図/d_bank_inf_change.pdf}
  \caption{d\_bank\_inf\_change.dart の状態遷移図}
  \label{fig:d_bank_inf_change}
\end{figure}

\newpage

%d_home.dart
\begin{table}[H]
\centering
%\caption{}
\small
\begin{tabular}{|c|l|p{0.4\textwidth}|}
\hline
モジュール名 &\multicolumn{2}{l|}{d\_job\_select.dart}\\ \hline
モジュール概要 &\multicolumn{2}{l|}{求人検索一覧画面を表示するモジュール}  \\ \hline
責任者 &\multicolumn{2}{l|}{金澤優} \\ \hline
\multirow{9}{*}{構成要素} & クラス名 & DJobRouterPage
\\ \cline{2-3}
& クラス種類 & extends AppRouter\\ \cline{2-3}
& 処理概要 & JobをアクセスするためのRouter&  \\ \cline{2-3}
& 入力 & なし \\ \cline{2-3}
& 出力 & Router \\ \cline{2-3}
&\multicolumn{2}{l|}{他クラスとの関係} \\\cline{2-3}
&\multicolumn{2}{p{0.9\linewidth}|}{なし} \\ \hline

\multirow{9}{*}{構成要素} & クラス名 & DJobSelect \\ \cline{2-3}
& クラス種類 & extends StatelessWidget \\ \cline{2-3}
& 処理概要 & 求人検索一覧画面を動的に変更するクラス\\ \cline{2-3}
& 入力 & なし \\ \cline{2-3}
& 出力 & Widget \\ \cline{2-3}
&\multicolumn{2}{l|}{他クラスとの関係} \\\cline{2-3}
&\multicolumn{2}{p{0.9\linewidth}|}{d_job_select.dartのDJobSelectStateクラスを組み込む} \\ \hline

\multirow{9}{*}{構成要素} & クラス名 & DJobSelectState \\ \cline{2-3}
& クラス種類 & extends StatelessWidget \\ \cline{2-3}
& 処理概要 & 求人検索一覧画面の状態を返すクラス\\ \cline{2-3}
& 入力 & なし \\ \cline{2-3}
& 出力 & State \\ \cline{2-3}
&\multicolumn{2}{l|}{他クラスとの関係} \\\cline{2-3}
&\multicolumn{2}{p{0.9\linewidth}|}{post list.dart の PostList クラスを組み込む

else list.dart の ElseList クラスを呼び出すことでおすすめ、商品数、地域、一覧画面を表示
search post.dart の SearchPost を呼び出すことで検索結果画面を表示

job_select_detail.dart の JobSelectDetailクラスを呼び出すことで求人詳細画面を表示} \\ \hline

\end{tabular}
\end{table}

\begin{figure}[H]
  \centering
  \includegraphics[width=0.8\textwidth]{./状態遷移図/delivery_状態遷移図/d_job_select.pdf}
  \caption{d\_job\_select.dart の状態遷移図}
  \label{fig:d_job_select}
\end{figure}


\newpage

%d_home.dart
\begin{table}[H]
\centering
%\caption{}
\small
\begin{tabular}{|c|l|p{0.4\textwidth}|}
\hline
モジュール名 &\multicolumn{2}{l|}{d\_job\_detail.dart}\\ \hline
モジュール概要 &\multicolumn{2}{l|}{求人情報詳細画面を表示するモジュール}  \\ \hline
責任者 &\multicolumn{2}{l|}{金澤優} \\ \hline
\multirow{9}{*}{構成要素} & クラス名 & DJobDetail
\\ \cline{2-3}
& クラス種類 & extends StatelessWidget\\ \cline{2-3}
& 処理概要 & 求人情報詳細画面を動的に変更するクラス \\ \cline{2-3}
& 入力 & なし \\ \cline{2-3}
& 出力 & Widget \\ \cline{2-3}
&\multicolumn{2}{l|}{他クラスとの関係} \\\cline{2-3}
&\multicolumn{2}{p{0.9\linewidth}|}{d_job_detail.dartのDJobDetailStateクラスを組み込む} \\ \hline

\multirow{9}{*}{構成要素} & クラス名 & DJobDetailState \\ \cline{2-3}
& クラス種類 & extends StatelessWidget \\ \cline{2-3}
& 処理概要 & 求人情報画面の状態を返すクラス\\ \cline{2-3}
& 入力 & なし \\ \cline{2-3}
& 出力 & State \\ \cline{2-3}
&\multicolumn{2}{l|}{他クラスとの関係} \\\cline{2-3}
&\multicolumn{2}{p{0.9\linewidth}|}{change general corporation.dart の ChangeGeneralCorporation プロバイダを組み込む

title appbar.dart の TitleAppBar クラスを組み込む

general mem inf conf.dart のGeneralMemInfConfクラスを呼び出し一般会員情報確認画面を表示

戻るボタンの選択で前のモジュールに戻る} \\ \hline
\end{tabular}
\end{table}

%上の他クラス・関数との関係を直す

\begin{figure}[H]
  \centering
  \includegraphics[width=0.8\textwidth]{./状態遷移図/delivery_状態遷移図/d_job_detail.pdf}
  \caption{d\_job\_detail.dart の状態遷移図}
  \label{fig:d_job_detail}
\end{figure}

\newpage

%d_home.dart
\begin{table}[H]
\centering
%\caption{}
\small
\begin{tabular}{|c|l|p{0.4\textwidth}|}
\hline
モジュール名 &\multicolumn{2}{l|}{d\_map.dart}\\ \hline
モジュール概要 &\multicolumn{2}{l|}{求人情報より配達先ルートを示したマップを表示するモジュール}  \\ \hline
責任者 &\multicolumn{2}{l|}{金澤優} \\ \hline
\multirow{9}{*}{構成要素} & クラス名 & Dmap
\\ \cline{2-3}
& クラス種類 & extends StatelessWidget\\ \cline{2-3}
& 処理概要 & 配達先ルート画面を動的に変更するクラス \\ \cline{2-3}
& 入力 & なし \\ \cline{2-3}
& 出力 & Widget \\ \cline{2-3}
&\multicolumn{2}{l|}{他クラスとの関係} \\\cline{2-3}
&\multicolumn{2}{p{0.9\linewidth}|}{d_job_detail.dartのDJobDetailStateクラスを組み込む} \\ \hline

\multirow{9}{*}{構成要素} & クラス名 & DMapDetailState \\ \cline{2-3}
& クラス種類 & extends StatelessWidget \\ \cline{2-3}
& 処理概要 & 配達先ルートの状態を返すクラス\\ \cline{2-3}
& 入力 & なし \\ \cline{2-3}
& 出力 & State \\ \cline{2-3}
&\multicolumn{2}{l|}{他クラスとの関係} \\\cline{2-3}
&\multicolumn{2}{p{0.9\linewidth}|}{change general corporation.dart の ChangeGeneralCorporation プロバイダを組み込む

title appbar.dart の TitleAppBar クラスを組み込む

general mem inf conf.dart のGeneralMemInfConfクラスを呼び出し一般会員情報確認画面を表示

戻るボタンの選択で前のモジュールに戻る} \\ \hline
\end{tabular}
\end{table}

%上の他クラス・関数との関係を直す

\begin{figure}[H]
  \centering
  \includegraphics[width=0.8\textwidth]{./状態遷移図/delivery_状態遷移図/d_map.pdf}
  \caption{d\_map.dart の状態遷移図}
  \label{fig:d_map}
\end{figure}

\newpage

%d_home.dart
\begin{table}[H]
\centering
%\caption{}
\small
\begin{tabular}{|c|l|p{0.4\textwidth}|}
\hline
モジュール名 &\multicolumn{2}{l|}{d\_job\_map\_select.dart}\\ \hline
モジュール概要 &\multicolumn{2}{l|}{マップにて配達員先住所(求人情報)を検索する画面を表示するモジュール}  \\ \hline
責任者 &\multicolumn{2}{l|}{金澤優} \\ \hline
\multirow{9}{*}{構成要素} & クラス名 & DJobMapSelect
\\ \cline{2-3}
& クラス種類 & extends StatelessWidget\\ \cline{2-3}
& 処理概要 & マップでの検索画面を動的に変更するクラス  \\ \cline{2-3}
& 入力 & なし \\ \cline{2-3}
& 出力 & Widget \\ \cline{2-3}
&\multicolumn{2}{l|}{他クラスとの関係} \\\cline{2-3}
&\multicolumn{2}{p{0.9\linewidth}|}{d_job_map_select.dartのDJobMapSelectStateクラスを組み込む} \\ \hline

\multirow{9}{*}{構成要素} & クラス名 & DJobMapSelectState \\ \cline{2-3}
& クラス種類 & extends StatelessWidget \\ \cline{2-3}
& 処理概要 & マップでの検索画面の状態を返すクラス\\ \cline{2-3}
& 入力 & なし \\ \cline{2-3}
& 出力 & Widget \\ \cline{2-3}
&\multicolumn{2}{l|}{他クラスとの関係} \\\cline{2-3}
&\multicolumn{2}{p{0.9\linewidth}|}{d_job_select.dartのDJobSelectStateクラスを組み込む} \\ \hline

\multirow{9}{*}{構成要素} & クラス名 & DJobSelectState \\ \cline{2-3}
& クラス種類 & extends StatelessWidget \\ \cline{2-3}
& 処理概要 & マップでの検索画面の状態を返すクラス\\ \cline{2-3}
& 入力 & なし \\ \cline{2-3}
& 出力 & State \\ \cline{2-3}
&\multicolumn{2}{l|}{他クラスとの関係} \\\cline{2-3}
&\multicolumn{2}{p{0.9\linewidth}|}{戻るボタンの選択で前のモジュールに戻る
job_list.dartのJobListクラスを読み込む
} \\ \hline

\end{tabular}
\end{table}

\begin{figure}[H]
  \centering
  \includegraphics[width=0.8\textwidth]{./状態遷移図/delivery_状態遷移図/d_job_map_select.pdf}
  \caption{d\_job\_map\_select.dart の状態遷移図}
  \label{fig:d_job_select}
\end{figure}


\subsection{provider}


\subsection{component}
\begin{table}[H]
\centering
%\caption{}
\small
\begin{tabular}{|c|l|p{0.4\textwidth}|}
\hline
モジュール名 &\multicolumn{2}{l|}{overscreen\_controller.dart}\\ \hline
モジュール概要 &\multicolumn{2}{l|}{オーバーレイの表示、非表示の操作を行うコンポーネント}  \\ \hline
責任者 &\multicolumn{2}{l|}{梶本和希} \\ \hline
\multirow{9}{*}{構成要素} & クラス名 & OverScreenController\\ \cline{2-3}
& クラス種類 & immutable \\ \cline{2-3}
& 処理概要 & オーバーレイの内容を返す \\ \cline{2-3}
& 入力 & なし \\ \cline{2-3}
& 出力 & Widget \\ \cline{2-3}
&\multicolumn{2}{l|}{他クラスとの関係} \\\cline{2-3}

&\multicolumn{2}{p{0.9\linewidth}|}{CloseLoadingScreen によってオーバレイを閉じる
UpdateLoadingScreen によってオーバーレイを表示} \\ \hline
\end{tabular}
\end{table}

\begin{table}[H]
\centering
%\caption{}
\small
\begin{tabular}{|c|l|p{0.4\textwidth}|}
\hline
モジュール名 &\multicolumn{2}{l|}{finishscreen.dart}\\ \hline
モジュール概要 &\multicolumn{2}{l|}{完了画面を生成するコンポーネント}  \\ \hline
責任者 &\multicolumn{2}{l|}{梶本和希} \\ \hline
\multirow{9}{*}{構成要素} & クラス名 & FinishScreen \\ \cline{2-3}
& クラス種類 & extendsStatelessWidget \\ \cline{2-3}
& 処理概要 & アイコン、テキストを引数で与えることによって、任意の完了画面を生成するコンポーネント \\ \cline{2-3}
& 入力 & StringappbarText, IconDataappIcon,Stringtext,StringbuttonTextbool isAppbar \\ \cline{2-3}
& 出力 & Widget \\ \cline{2-3}
&\multicolumn{2}{l|}{他クラスとの関係} \\\cline{2-3}

&\multicolumn{2}{p{0.9\linewidth}|}{titleappbar.dartのTitleAppBarクラスを組み込む
normalbottomappbar.dartのNormalBottomAppBarクラスを組み込む
buttonset.dartのButtonSetクラスを組み込む
LoginRouteの最初の画面へと戻る} \\ \hline
\end{tabular}
\end{table}

%画像の追加
\begin{figure}[H]
  \centering
  \safeincludegraphics[width=0.75\textwidth]{状態遷移図/common_状態遷移図/FinishScreen.pdf}
  \caption{FinishScreenの状態遷移図}
  \label{fig:finishscreen.dart}
\end{figure}

\begin{table}[H]
\centering
%\caption{}
\small
\begin{tabular}{|c|l|p{0.4\textwidth}|}
\hline
モジュール名 &\multicolumn{2}{l|}{botton\_set.dart}\\ \hline
モジュール概要 &\multicolumn{2}{l|}{任意のテキストが書かれたボタンを生成するモジュール}  \\ \hline
責任者 &\multicolumn{2}{l|}{梶本和希} \\ \hline
\multirow{9}{*}{構成要素} & クラス名 & ButtonSet \\ \cline{2-3}
& クラス種類 &  extendsStatelessWidget\\ \cline{2-3}
& 処理概要 & bodyの呼び出された部分にボタンを設置するコンポーネント \\ \cline{2-3}
& 入力 & Sting buttonName \\ \cline{2-3}
& 出力 & Widget \\ \cline{2-3}
&\multicolumn{2}{l|}{他クラスとの関係} \\\cline{2-3}

&\multicolumn{2}{p{0.9\linewidth}|}{change\_user\_role.dart のChangeUserRole.dartクラスを組み込む} \\ \hline
\end{tabular}
\end{table}

\begin{table}[H]
\centering
%\caption{}
\small
\begin{tabular}{|c|l|p{0.4\textwidth}|}
\hline
モジュール名 &\multicolumn{2}{l|}{title\_appbar.dart}\\ \hline
モジュール概要 &\multicolumn{2}{l|}{AppBarの下にページタイトルを表示するAppBarを作成するコンポーネント}  \\ \hline
責任者 &\multicolumn{2}{l|}{梶本和希} \\ \hline
\multirow{9}{*}{構成要素} & クラス名 & TitleAppBar \\ \cline{2-3}
& クラス種類 & extendsStatelessWidget \\ \cline{2-3}
& 処理概要 & AppBarのbottomにてもう一度AppBarを作成し、そこにページのタイトルを
表示させるAppBarを生成するコンポーネント \\ \cline{2-3}
& 入力 & String appbarText, bool isReturn \\ \cline{2-3}
& 出力 & Widget \\ \cline{2-3}
&\multicolumn{2}{l|}{他クラスとの関係} \\\cline{2-3}

&\multicolumn{2}{p{0.9\linewidth}|}{pop を行う} \\ \hline
\end{tabular}
\end{table}

%画像の追加
\begin{figure}[H]
  \centering
  \safeincludegraphics[width=0.75\textwidth]{状態遷移図/common_状態遷移図/TitleAppbar.pdf}
  \caption{TitleAppbarの状態遷移図}
  \label{fig:TitleAppbar.dart}
\end{figure}


\begin{table}[H]
\centering
%\caption{}
\small
\begin{tabular}{|c|l|p{0.4\textwidth}|}
\hline
モジュール名 &\multicolumn{2}{l|}{normal\_bottom\_appbar.dart}\\ \hline
モジュール概要 &\multicolumn{2}{l|}{BottomNavigationBarがない画面で用いるBottomAppBarを作成するコンポーネント}  \\ \hline
責任者 &\multicolumn{2}{l|}{梶本和希} \\ \hline
\multirow{9}{*}{構成要素} & クラス名 & NormalBottomAppBar \\ \cline{2-3}
& クラス種類 & extendsStatelessWidget \\ \cline{2-3}
& 処理概要 & 色がついただけのBottomAppBarのWidgetを返すコンポーネント \\ \cline{2-3}
& 入力 & なし \\ \cline{2-3}
& 出力 & Widget \\ \cline{2-3}
&\multicolumn{2}{l|}{他クラスとの関係} \\\cline{2-3}

&\multicolumn{2}{p{0.9\linewidth}|}{なし} \\ \hline
\end{tabular}
\end{table}
