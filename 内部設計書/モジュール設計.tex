\subsection{overlay}

\subsubsection{共通機能}

\begin{table}[h]
\centering
%\caption{}
\small
\begin{tabular}{|c|l|p{0.4\textwidth}|}
\hline
モジュール名 &\multicolumn{2}{l|}{withdraw.dart}\\ \hline
モジュール概要 &\multicolumn{2}{l|}{退会を確定するかどうかを確認するオーバーレイ	}  \\ \hline
責任者 &\multicolumn{2}{l|}{梶本和希} \\ \hline
\multirow{9}{*}{構成要素} & クラス名 & Withdraw\\ \cline{2-3}
& クラス種類 & OverEntry \\ \cline{2-3}
& 処理概要 & Overlayの内容を返す \\ \cline{2-3}
& 入力 & Buildcontext context \\ \cline{2-3}
& 出力 & Maaterial \\ \cline{2-3}
&\multicolumn{2}{l|}{他クラスとの関係} \\\cline{2-3}

&\multicolumn{2}{p{0.9\linewidth}|}{withdraw.dart の Withdraw().show(context) を呼び出すとオーバレイ表示

キャンセルが押されるとWithdraw().hide() によりcontrolle.close が呼び出されオーバレイを閉じる

退会を押すと退会処理後、Withdraw().hide() によりcontrolle.close が呼び出されオーバレイを閉じる} \\ \hline
\end{tabular}
\end{table}


%画像の追加

\begin{table}[h]
\centering
%\caption{}
\small
\begin{tabular}{|c|l|p{0.4\textwidth}|}
\hline
モジュール名 &\multicolumn{2}{l|}{logout.dart}\\ \hline
モジュール概要 &\multicolumn{2}{l|}{ログアウトを確定するかどうかを確認するオーバーレイ	}  \\ \hline
責任者 &\multicolumn{2}{l|}{岡林磨目} \\ \hline
\multirow{9}{*}{構成要素} & クラス名 & Logout\\ \cline{2-3}
& クラス種類 & OverEntry\\ \cline{2-3}
& 処理概要 & Overlayの内容を返す \\ \cline{2-3}
& 入力 & Buildcontext context\\ \cline{2-3}
& 出力 & Material\\ \cline{2-3}
&\multicolumn{2}{l|}{他クラスとの関係} \\\cline{2-3}
&\multicolumn{2}{p{0.9\linewidth}|}{logout.dartのLogout().show(context)を呼び出すとオーバーレイ表示

キャンセルを押すと Logout().hide() によりcontrolle.close が呼び出されオーバレイを閉じる

ログアウトを押すことで login\_page.dart の LoginPage クラスを呼び出すことで、ログイン画面を表示} \\ \hline
\end{tabular}
\end{table}


%画像の追加

\begin{table}[h]
\centering
%\caption{}
\small
\begin{tabular}{|c|l|p{0.4\textwidth}|}
\hline
モジュール名 &\multicolumn{2}{l|}{rule\_screen.dart}\\ \hline
モジュール概要 &\multicolumn{2}{l|}{利用規約を表示するオーバーレイ}  \\ \hline
責任者 &\multicolumn{2}{l|}{岡林磨目} \\ \hline
\multirow{9}{*}{構成要素} & クラス名 & RuleScreen\\ \cline{2-3}
& クラス種類 & OverEntry \\ \cline{2-3}
& 処理概要 & Overlayの内容を返す \\ \cline{2-3}
& 入力 & Buildcontext context \\ \cline{2-3}
& 出力 & Maaterial \\ \cline{2-3}
&\multicolumn{2}{l|}{他クラスとの関係} \\\cline{2-3}

&\multicolumn{2}{p{0.9\linewidth}|}{rule\_screen.dart の RuleScreen().show(context)を呼び出すとオーバーレイ表示

閉じるを押すとRuleScreen().hide()により controlle.closeが呼び出されオーバレイを閉じる} \\ \hline
\end{tabular}
\end{table}

%画像の追加

\begin{table}[h]
\centering
%\caption{}
\small
\begin{tabular}{|c|l|p{0.4\textwidth}|}
\hline
モジュール名 &\multicolumn{2}{l|}{cancel\_inf\_change.dart}\\ \hline
モジュール概要 &\multicolumn{2}{l|}{情報の変更をキャンセルするかを確認するオーバーレイ}  \\ \hline
責任者 &\multicolumn{2}{l|}{岡林磨目} \\ \hline
\multirow{9}{*}{構成要素} & クラス名 & CancelInfChange\\ \cline{2-3}
& クラス種類 & OverEntry \\ \cline{2-3}
& 処理概要 & Overlayの内容を返す \\ \cline{2-3}
& 入力 & Buildcontext context \\ \cline{2-3}
& 出力 & Maaterial \\ \cline{2-3}
&\multicolumn{2}{l|}{他クラスとの関係} \\\cline{2-3}

&\multicolumn{2}{p{0.9\linewidth}|}{cancel\_inf\_change.dartの CancelInfChange().show(context) を呼び出すとオーバレイ表示

いいえを押すとCancelInfChange().hide() によりcontrolle.closeが呼び出されオーバレイを閉じる

はいを押すと変更キャンセル処理後、CancelInfChange().hide()によりcontrolle.closeが呼び出されオーバレイを閉じる} \\ \hline
\end{tabular}
\end{table}

%画像の追加


\begin{table}[h]
\centering
%\caption{}
\small
\begin{tabular}{|c|l|p{0.4\textwidth}|}
\hline
モジュール名 &\multicolumn{2}{l|}{confirm\_inf\_change.dart}\\ \hline
モジュール概要 &\multicolumn{2}{l|}{情報の変更を確定するかを確認するオーバーレイ}  \\ \hline
責任者 &\multicolumn{2}{l|}{岡林磨目} \\ \hline
\multirow{9}{*}{構成要素} & クラス名 & ConfirmInfChange\\ \cline{2-3}
& クラス種類 & OverEntry \\ \cline{2-3}
& 処理概要 & Overlayの内容を返す \\ \cline{2-3}
& 入力 & Buildcontext context \\ \cline{2-3}
& 出力 & Maaterial \\ \cline{2-3}
&\multicolumn{2}{l|}{他クラスとの関係} \\\cline{2-3}

&\multicolumn{2}{p{0.9\linewidth}|}{confirm\_inf\_change.dart の ConfirmInfChange().show(context) を呼び出すとオーバレイ表示

いいえを押すと ConfirmInfChange().hide() により controlle.close が呼び出されオーバレイを閉じる

はいを押すと変更確定処理後、ConfirmInfChange().hide() により controlle.close が呼び出されオーバレイを閉じる} \\ \hline
\end{tabular}
\end{table}

\subsubsection{配達員}

\begin{table}[h]
\centering
%\caption{}
\small
\begin{tabular}{|c|l|p{0.4\textwidth}|}
\hline
モジュール名 &\multicolumn{2}{l|}{rule\_screen.dart}\\ \hline
モジュール概要 &\multicolumn{2}{l|}{利用規約を表示するオーバーレイ}  \\ \hline
責任者 &\multicolumn{2}{l|}{岡林磨目} \\ \hline
\multirow{9}{*}{構成要素} & クラス名 & RuleScreen\\ \cline{2-3}
& クラス種類 & OverEntry \\ \cline{2-3}
& 処理概要 & Overlayの内容を返す \\ \cline{2-3}
& 入力 & Buildcontext context \\ \cline{2-3}
& 出力 & Maaterial \\ \cline{2-3}
&\multicolumn{2}{l|}{他クラスとの関係} \\\cline{2-3}

&\multicolumn{2}{p{0.9\linewidth}|}{rule\_screen.dart の RuleScreen().show(context)を呼び出すとオーバーレイ表示

閉じるを押すとRuleScreen().hide()により controlle.closeが呼び出されオーバレイを閉じる} \\ \hline
\end{tabular}
\end{table}

\begin{figure}[H]
  \centering
  \includegraphics[width=0.5\textwidth]{./状態遷移図/delivery_状態遷移図/Notice_detail.pdf}
  \caption{配達員アカウント管理画面}
  \label{fig:a配達員アカウント管理画面}
\end{figure}
\subsection{provider}

%画像の追加
\subsection{provider}

\begin{table}[h]
\centering
%\caption{}
\small
\begin{tabular}{|c|l|p{0.4\textwidth}|}
\hline
モジュール名 &\multicolumn{2}{l|}{change\_user\_role.dart}\\ \hline
モジュール概要 &\multicolumn{2}{l|}{アプリ全体で使用するプロバイダモジュール	}  \\ \hline
責任者 &\multicolumn{2}{l|}{岡林磨目} \\ \hline
\multirow{9}{*}{構成要素} & クラス名 & ChangeUserRole\\ \cline{2-3}
& クラス種類 & ChangeNotifier \\ \cline{2-3}
& 処理概要 & アプリ全体で使用する依頼側・配達側・店舗側の判断を行う int 型変数を定義している。
また、値が変更されると、それに対応した色(UI)へ変更が行われる。 \\ \cline{2-3}
& 入力 & なし \\ \cline{2-3}
& 出力 & なし \\ \cline{2-3}
&\multicolumn{2}{l|}{他クラスとの関係} \\\cline{2-3}

&\multicolumn{2}{p{0.9\linewidth}|}{なし} \\ \hline
\end{tabular}
\end{table}

%画像の追加
\subsection{page}
\subsubsection{共通}


\begin{table}[h]
\centering
%\caption{}
\small
\begin{tabular}{|c|l|p{0.4\textwidth}|}
\hline
モジュール名 &\multicolumn{2}{l|}{change\_root\_page.dart}\\ \hline
モジュール概要 &\multicolumn{2}{l|}{アプリ全体を包括するモジュール}  \\ \hline
責任者 &\multicolumn{2}{l|}{岡林磨目} \\ \hline
\multirow{9}{*}{構成要素} & クラス名 & ChangeRootPage\\ \cline{2-3}
& クラス種類 & extends StatelessWidget \\ \cline{2-3}
& 処理概要 & アプリの大まかな区別である LoginRoute と RootRoute を包括しているクラス \\ \cline{2-3}
& 入力 & なし \\ \cline{2-3}
& 出力 & Widget \\ \cline{2-3}
&\multicolumn{2}{l|}{他クラスとの関係} \\\cline{2-3}

&\multicolumn{2}{p{0.9\linewidth}|}{app\_router.gr.dart の LoginRoute クラスを組み込む} \\ \hline
\end{tabular}
\end{table}

%画像の追加

\begin{table}[h]
\centering
%\caption{}
\small
\begin{tabular}{|c|l|p{0.4\textwidth}|}
\hline
モジュール名 &\multicolumn{2}{l|}{new\_member.dart}\\ \hline
モジュール概要 &\multicolumn{2}{l|}{新規会員登録画面を生成するモジュール}  \\ \hline
責任者 &\multicolumn{2}{l|}{梶本和希} \\ \hline
\multirow{9}{*}{構成要素} & クラス名 & NewMember\\ \cline{2-3}
& クラス種類 & extends StatelessWidget \\ \cline{2-3}
& 処理概要 & 新規会員登録画面のWidgetを動的に変更するクラス \\ \cline{2-3}
& 入力 & なし \\ \cline{2-3}
& 出力 & Widget \\ \cline{2-3}
&\multicolumn{2}{l|}{他クラスとの関係} \\\cline{2-3}
&\multicolumn{2}{p{0.9\linewidth}|}{new\_member.dart の NewMenberState クラスを組み込む} \\ \hline

\multirow{9}{*}{構成要素} & クラス名 & NewMemberState\\ \cline{2-3}
& クラス種類 & extends StatelessWidget \\ \cline{2-3}
& 処理概要 & 新規会員登録画面を表示するクラス \\ \cline{2-3}
& 入力 & なし \\ \cline{2-3}
& 出力 & Widget \\ \cline{2-3}
&\multicolumn{2}{l|}{他クラスとの関係} \\\cline{2-3}
&\multicolumn{2}{p{0.9\linewidth}|}{rule\_screen.dart の RuleScreen クラスを呼び出すことで利用規約を表示

login\_page.dart のLoginPageクラスを呼び出しログイン画面を表示

戻るボタンの選択で前のモジュールに戻る} \\ \hline


\end{tabular}
\end{table}

%画像の追加
\begin{table}[h]
\centering
%\caption{}
\small
\begin{tabular}{|c|l|p{0.4\textwidth}|}
\hline
モジュール名 &\multicolumn{2}{l|}{mypage.dart}\\ \hline
モジュール概要 &\multicolumn{2}{l|}{マイページ画面を表示するモジュール}  \\ \hline
責任者 &\multicolumn{2}{l|}{岡林磨目} \\ \hline
\multirow{9}{*}{構成要素} & クラス名 & MyPageRouterPage \\ \cline{2-3}
& クラス種類 & extends AppRouter \\ \cline{2-3}
& 処理概要 & MyPage にアクセスするための Router \\ \cline{2-3}
& 入力 & なし \\ \cline{2-3}
& 出力 & Route \\ \cline{2-3}
&\multicolumn{2}{l|}{他クラスとの関係} \\\cline{2-3}
&\multicolumn{2}{p{0.9\linewidth}|}{なし} \\ \hline

\multirow{9}{*}{構成要素} & クラス名 & MyPage\\ \cline{2-3}
& クラス種類 & extends StatelessWidget \\ \cline{2-3}
& 処理概要 & マイページ画面を動的に変更するクラス \\ \cline{2-3}
& 入力 & なし \\ \cline{2-3}
& 出力 & Widget \\ \cline{2-3}
&\multicolumn{2}{l|}{他クラスとの関係} \\\cline{2-3}
&\multicolumn{2}{p{0.9\linewidth}|}{MyPage.dart の MyPageState クラスを組み込む} \\ \hline


\multirow{9}{*}{構成要素} & クラス名 & MyPageState\\ \cline{2-3}
& クラス種類 & extends StatelessWidget \\ \cline{2-3}
& 処理概要 & マイページ画面の状態を返すクラス \\ \cline{2-3}
& 入力 & なし \\ \cline{2-3}
& 出力 & State \\ \cline{2-3}
&\multicolumn{2}{l|}{他クラスとの関係} \\\cline{2-3}
&\multicolumn{2}{p{0.9\linewidth}|}{title\_appbar.dart の TitleAppBar クラスを組み込む

mypage.dart の ScrollMyPageDetail クラスを組み込む} \\ \hline

\multirow{9}{*}{構成要素} & クラス名 & \\ \cline{2-3}
& クラス種類 &  \\ \cline{2-3}
& 処理概要 &  \\ \cline{2-3}
& 入力 &  \\ \cline{2-3}
& 出力 &  \\ \cline{2-3}
&\multicolumn{2}{l|}{他クラスとの関係} \\\cline{2-3}
&\multicolumn{2}{p{0.9\linewidth}|}{} \\ \hline

\multirow{9}{*}{構成要素} & クラス名 & \\ \cline{2-3}
& クラス種類 &  \\ \cline{2-3}
& 処理概要 &  \\ \cline{2-3}
& 入力 &  \\ \cline{2-3}
& 出力 &  \\ \cline{2-3}
&\multicolumn{2}{l|}{他クラスとの関係} \\\cline{2-3}
&\multicolumn{2}{p{0.9\linewidth}|}{} \\ \hline


\end{tabular}
\end{table}

%画像の追加
\begin{table}[h]
\centering
%\caption{}
\small
\begin{tabular}{|c|l|p{0.4\textwidth}|}
\hline
モジュール名 &\multicolumn{2}{l|}{login\_page.dart}\\ \hline
モジュール概要 &\multicolumn{2}{l|}{ログイン画面を生成するモジュール}  \\ \hline
責任者 &\multicolumn{2}{l|}{梶本和希} \\ \hline
\multirow{9}{*}{構成要素} & クラス名 & LoginPage\\ \cline{2-3}
& クラス種類 & extends StatelessWidget \\ \cline{2-3}
& 処理概要 & ログイン画面のWidgetを動的に変更するクラス \\ \cline{2-3}
& 入力 & なし \\ \cline{2-3}
& 出力 & Widget \\ \cline{2-3}
&\multicolumn{2}{l|}{他クラスとの関係} \\\cline{2-3}
&\multicolumn{2}{p{0.9\linewidth}|}{login\_page.dartのLoginPageStateクラスを組み込む} \\ \hline

\multirow{9}{*}{構成要素} & クラス名 & LoginPageState \\ \cline{2-3}
& クラス種類 & extends StatelessWidget \\ \cline{2-3}
& 処理概要 & ログイン画面の状態を返すクラス \\ \cline{2-3}
& 入力 & なし \\ \cline{2-3}
& 出力 & State \\ \cline{2-3}
&\multicolumn{2}{l|}{他クラスとの関係} \\\cline{2-3}
&\multicolumn{2}{p{0.9\linewidth}|}{new\_member.dartのNewMemberクラスを呼び出すことで、新規会員登録画面を表示

login\_page.dartのLoginAppBarクラスを組み込む} \\ \hline


\multirow{9}{*}{構成要素} & クラス名 & LoginAppBar\\ \cline{2-3}
& クラス種類 & extends StatelessWidget,imprementsPrefferdSizeWidget \\ \cline{2-3}
& 処理概要 & ログイン画面用に作成を行ったAppBarを生成するコンポーネント \\ \cline{2-3}
& 入力 & ChangeGeneralCorporationstore \\ \cline{2-3}
& 出力 & Widget \\ \cline{2-3}
&\multicolumn{2}{l|}{他クラスとの関係} \\\cline{2-3}
&\multicolumn{2}{p{0.9\linewidth}|}{なし} \\ \hline

\end{tabular}
\end{table}

%画像の追加