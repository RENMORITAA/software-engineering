以下に、本アプリケーションにおける主要なデータベース設計を示す。

\subsection{テーブル一覧}

\begin{table}[H]
\centering
\caption{テーブル一覧}
\label{tab:db_list}
\begin{tabularx}{\textwidth}{|l|X|}
\hline
\rowcolor[RGB]{230,230,230} \textbf{テーブル名} & \textbf{説明} \\ \hline
users & 全ユーザ(依頼者・配達員・店舗・管理者)の基底情報 \\ \hline
requester\_profiles & 依頼者の詳細情報(住所、クレジットカード情報等) \\ \hline
deliverer\_profiles & 配達員の詳細情報(履歴書、稼働状況、口座情報) \\ \hline
store\_profiles & 店舗の詳細情報(営業許可証、営業時間、口座情報) \\ \hline
products & 商品情報 \\ \hline
orders & 注文情報のヘッダー(合計金額、ステータス、依頼者ID) \\ \hline
order\_details & 注文明細(商品ID、数量) \\ \hline
deliveries & 配達情報(担当配達員ID、位置情報ログ) \\ \hline
notifications & 通知情報 \\ \hline
\end{tabularx}
\end{table}

\subsection{主要テーブル詳細}

\subsubsection*{users テーブル}
\begin{table}[H]
\centering
\begin{tabular}{|l|l|l|l|}
\hline
\rowcolor[RGB]{240,240,240} \textbf{カラム名} & \textbf{制約} & \textbf{型} & \textbf{説明} \\ \hline
id & PK & int & ユーザーID \\ \hline
email & NN, UQ & varchar & メールアドレス \\ \hline
password & NN & varchar & ハッシュ化されたパスワード \\ \hline
role & NN & char & 役割(requester/deliverer/store/admin) \\ \hline
created\_at & & datetime & 作成日時 \\ \hline
\end{tabular}
\end{table}

\subsubsection*{orders テーブル}
\begin{table}[H]
\centering
\begin{tabular}{|l|l|l|l|}
\hline
\rowcolor[RGB]{240,240,240} \textbf{カラム名} & \textbf{制約} & \textbf{型} & \textbf{説明} \\ \hline
id & PK & int & 注文ID \\ \hline
requester\_id & FK & int & 依頼者ID \\ \hline
store\_id & FK & int & 店舗ID \\ \hline
deliverer\_id & FK & int & 配達員ID(受注後設定) \\ \hline
status & NN & varchar & 注文ステータス \\ \hline
total\_price & NN & int & 合計金額 \\ \hline
ordered\_at & & datetime & 注文日時 \\ \hline
\end{tabular}
\end{table}

\subsubsection*{deliveries テーブル}
\begin{table}[H]
\centering
\begin{tabular}{|l|l|l|l|}
\hline
\rowcolor[RGB]{240,240,240} \textbf{カラム名} & \textbf{制約} & \textbf{型} & \textbf{説明} \\ \hline
id & PK & int & 配達ID \\ \hline
order\_id & FK & int & 注文ID \\ \hline
pickup\_time & & datetime & 商品受け取り時間 \\ \hline
delivery\_time & & datetime & 配達完了時間 \\ \hline
status & NN & varchar & 配達状況 \\ \hline
\end{tabular}
\end{table}