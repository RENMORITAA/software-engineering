\documentclass{jlreq} 
%\usepackage{url} 
\usepackage{float}
\usepackage{graphicx} 
\usepackage{multicol}
\usepackage{ascmac}
\usepackage{siunitx}
\usepackage{listings}
\usepackage{xcolor}
\usepackage{amsmath}
%赤い枠が出ないようにhyperrefを読み込む 

\usepackage{fancyvrb}
\usepackage{hyperref}%pdfに赤い枠を出さない
\hypersetup{
  colorlinks=true,
  linkcolor=black,
  filecolor=magenta,
  urlcolor=blue,
  citecolor=black,
}

\lstnewenvironment{mylisting}[1][]%
  {\lstset{
    frame=single,
    basicstyle=\ttfamily\small,
    numbersep=6pt,
    tabsize=3,
    extendedchars=true,
    xleftmargin=17pt,
    framexleftmargin=17pt,
    breaklines=true,
    numbers=left,
    language=Matlab,  % ← ここで言語指定
    #1
  }}{}


\begin{document}

% --- 表紙 ---
\begin{titlepage}
  \centering
  \vspace{2cm}
  {\LARGE \bfseries 外部設計書 \par}
  \vspace{1cm}
  {\LARGE \bfseries 香美市特化型配達サービス \par}
  \vspace{1cm}
   {\LARGE \bfseries Stellar Delivery \par}
  \vspace{3cm}
  %第2版
  {\Large 第1版 \par}
  %少し開けて作成日を入れる
  \vspace{4cm}
  {\Large \today \par}
  \vspace{0.5cm}
  {\LARGE \bfseries StellarWorks \par}
  \vfill
\end{titlepage}

\newpage

\tableofcontents
\newpage


\section{はじめに}
本書では, 弊社がシステム提案書で提案した地域密着型アルバイト求人アプリ「REEL」の機能詳細について示
す. まず,本システムの概要について示す.なお,主張の根拠はシステム提案書を参照されたい.次に本システム
の画像遷移図とユーザーインターフェースを示す. その後, 運用・保守について示し, 最後に本システムを構築する
データベース設計とネットワーク設計について示す.

test

\section{システム概要}
test

\section{機能設計}
\subsection{機能概要図}




\subsection{共通機能}




\subsection{一般利用者}



\subsection{法人利用者}



\section{UI設計}
\subsection{一般利用者}



\subsection{法人利用者}

\section{運用保守設計}
\subsection{運用}

\subsection{保守}

\subsection{セキュリティ対策}

\subsubsection{ユーザー認証}

\subsubsection{パスワードのハッシュ化}

\subsubsection{SQLインジェクション対策}

\subsubsection{SSL通信}

\subsection{障害対策}

\subsubsection{ハードウェア障害}

\subsubsection{ソフトウェア障害}

\subsubsection{アプリケーション開発及びテストの障害対策}


\section{データベース設計}
test

\section{ネットワーク設計}
test

\end{document}