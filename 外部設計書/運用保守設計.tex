本アプリケーションのバックエンド側はAmazonWebServicesで構築する. したがって,24時間365日という
時間はオンライン/バッチを含みシステムが稼働している時間として定義する.
本アプリに関しての運用・保守について以下に記述する.
\subsection{運用}
運用事項について以下の項目を行う。
\begin{itemize}
    \item システムが正常に稼働しているかの監視
    \item データのバックアップ処理
    \item 障害対応
    \item セキュリティ対策
    \item 定期メンテナンス
    \item ユーザーからの問い合わせ対応
\end{itemize}
個人情報へのアクセス権は弊社社員6名に限定し,アクセスログを取得することで不正アクセスを防止する.
ユーザーからの問い合わせに関しては,専用の問い合わせフォームを設置し,対応状況を管理する.
\subsection{保守}
保守事項については以下の項目とする。
\begin{itemize}
    \item ソフトウェアのバージョンアップ
    \item バグ修正
    \item 機能追加・改善
\end{itemize}
上記を定期的に実施し,システムの安定性とユーザー満足度を向上させる.
システムの根幹にかかわるバグが発見された場合は、速やかにシステムを停止し、原因の調査と修正を行う。


\subsection{セキュリティ対策}

\subsubsection{ユーザー認証}
ユーザー登録の際、メールで本人確認を行う。また、パスワードは強力なものを要求する。

\subsubsection{パスワードのハッシュ化}
パスワードは、安全なハッシュ関数を用いてハッシュ化し、データベースに保存する。
この操作を行うことで、万が一データベースが漏洩した場合でも、パスワードの平文が流出するリスクを軽減する。
また、個人情報にアクセスできる社員もパスワードを閲覧することができないようになる。
\subsubsection{SQLインジェクション対策}
SQLインジェクション攻撃を防ぐために、プリペアドステートメントを使用し、ユーザーからの入力を適切にエスケープする。
\subsubsection{SSL通信}
SSL通信を採用し、通信内容の盗聴や改ざんを防止する。
\subsection{障害対策}

\subsubsection{ハードウェア障害}
AWS上にバックエンドを構築することで、ハードウェア障害に対する冗長化を実現する。
AWSの障害以外では、ハードウェア障害は発生しないものとする。
\subsubsection{ソフトウェア障害}
ソフトウェア障害に対しては、定期的なバックアップと監視システムを導入することで、障害発生時に迅速に対応できる体制を整える。
\subsubsection{アプリケーション開発及びテストの障害対策}
開発者各々の環境に依存しないように、Dockerを用いて開発を行う。
